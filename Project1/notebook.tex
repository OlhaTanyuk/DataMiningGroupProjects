
% Default to the notebook output style

    


% Inherit from the specified cell style.




    
\documentclass[11pt]{article}

    
    
    \usepackage[T1]{fontenc}
    % Nicer default font (+ math font) than Computer Modern for most use cases
    \usepackage{mathpazo}

    % Basic figure setup, for now with no caption control since it's done
    % automatically by Pandoc (which extracts ![](path) syntax from Markdown).
    \usepackage{graphicx}
    % We will generate all images so they have a width \maxwidth. This means
    % that they will get their normal width if they fit onto the page, but
    % are scaled down if they would overflow the margins.
    \makeatletter
    \def\maxwidth{\ifdim\Gin@nat@width>\linewidth\linewidth
    \else\Gin@nat@width\fi}
    \makeatother
    \let\Oldincludegraphics\includegraphics
    % Set max figure width to be 80% of text width, for now hardcoded.
    \renewcommand{\includegraphics}[1]{\Oldincludegraphics[width=.8\maxwidth]{#1}}
    % Ensure that by default, figures have no caption (until we provide a
    % proper Figure object with a Caption API and a way to capture that
    % in the conversion process - todo).
    \usepackage{caption}
    \DeclareCaptionLabelFormat{nolabel}{}
    \captionsetup{labelformat=nolabel}

    \usepackage{adjustbox} % Used to constrain images to a maximum size 
    \usepackage{xcolor} % Allow colors to be defined
    \usepackage{enumerate} % Needed for markdown enumerations to work
    \usepackage{geometry} % Used to adjust the document margins
    \usepackage{amsmath} % Equations
    \usepackage{amssymb} % Equations
    \usepackage{textcomp} % defines textquotesingle
    % Hack from http://tex.stackexchange.com/a/47451/13684:
    \AtBeginDocument{%
        \def\PYZsq{\textquotesingle}% Upright quotes in Pygmentized code
    }
    \usepackage{upquote} % Upright quotes for verbatim code
    \usepackage{eurosym} % defines \euro
    \usepackage[mathletters]{ucs} % Extended unicode (utf-8) support
    \usepackage[utf8x]{inputenc} % Allow utf-8 characters in the tex document
    \usepackage{fancyvrb} % verbatim replacement that allows latex
    \usepackage{grffile} % extends the file name processing of package graphics 
                         % to support a larger range 
    % The hyperref package gives us a pdf with properly built
    % internal navigation ('pdf bookmarks' for the table of contents,
    % internal cross-reference links, web links for URLs, etc.)
    \usepackage{hyperref}
    \usepackage{longtable} % longtable support required by pandoc >1.10
    \usepackage{booktabs}  % table support for pandoc > 1.12.2
    \usepackage[inline]{enumitem} % IRkernel/repr support (it uses the enumerate* environment)
    \usepackage[normalem]{ulem} % ulem is needed to support strikethroughs (\sout)
                                % normalem makes italics be italics, not underlines
    

    
    
    % Colors for the hyperref package
    \definecolor{urlcolor}{rgb}{0,.145,.698}
    \definecolor{linkcolor}{rgb}{.71,0.21,0.01}
    \definecolor{citecolor}{rgb}{.12,.54,.11}

    % ANSI colors
    \definecolor{ansi-black}{HTML}{3E424D}
    \definecolor{ansi-black-intense}{HTML}{282C36}
    \definecolor{ansi-red}{HTML}{E75C58}
    \definecolor{ansi-red-intense}{HTML}{B22B31}
    \definecolor{ansi-green}{HTML}{00A250}
    \definecolor{ansi-green-intense}{HTML}{007427}
    \definecolor{ansi-yellow}{HTML}{DDB62B}
    \definecolor{ansi-yellow-intense}{HTML}{B27D12}
    \definecolor{ansi-blue}{HTML}{208FFB}
    \definecolor{ansi-blue-intense}{HTML}{0065CA}
    \definecolor{ansi-magenta}{HTML}{D160C4}
    \definecolor{ansi-magenta-intense}{HTML}{A03196}
    \definecolor{ansi-cyan}{HTML}{60C6C8}
    \definecolor{ansi-cyan-intense}{HTML}{258F8F}
    \definecolor{ansi-white}{HTML}{C5C1B4}
    \definecolor{ansi-white-intense}{HTML}{A1A6B2}

    % commands and environments needed by pandoc snippets
    % extracted from the output of `pandoc -s`
    \providecommand{\tightlist}{%
      \setlength{\itemsep}{0pt}\setlength{\parskip}{0pt}}
    \DefineVerbatimEnvironment{Highlighting}{Verbatim}{commandchars=\\\{\}}
    % Add ',fontsize=\small' for more characters per line
    \newenvironment{Shaded}{}{}
    \newcommand{\KeywordTok}[1]{\textcolor[rgb]{0.00,0.44,0.13}{\textbf{{#1}}}}
    \newcommand{\DataTypeTok}[1]{\textcolor[rgb]{0.56,0.13,0.00}{{#1}}}
    \newcommand{\DecValTok}[1]{\textcolor[rgb]{0.25,0.63,0.44}{{#1}}}
    \newcommand{\BaseNTok}[1]{\textcolor[rgb]{0.25,0.63,0.44}{{#1}}}
    \newcommand{\FloatTok}[1]{\textcolor[rgb]{0.25,0.63,0.44}{{#1}}}
    \newcommand{\CharTok}[1]{\textcolor[rgb]{0.25,0.44,0.63}{{#1}}}
    \newcommand{\StringTok}[1]{\textcolor[rgb]{0.25,0.44,0.63}{{#1}}}
    \newcommand{\CommentTok}[1]{\textcolor[rgb]{0.38,0.63,0.69}{\textit{{#1}}}}
    \newcommand{\OtherTok}[1]{\textcolor[rgb]{0.00,0.44,0.13}{{#1}}}
    \newcommand{\AlertTok}[1]{\textcolor[rgb]{1.00,0.00,0.00}{\textbf{{#1}}}}
    \newcommand{\FunctionTok}[1]{\textcolor[rgb]{0.02,0.16,0.49}{{#1}}}
    \newcommand{\RegionMarkerTok}[1]{{#1}}
    \newcommand{\ErrorTok}[1]{\textcolor[rgb]{1.00,0.00,0.00}{\textbf{{#1}}}}
    \newcommand{\NormalTok}[1]{{#1}}
    
    % Additional commands for more recent versions of Pandoc
    \newcommand{\ConstantTok}[1]{\textcolor[rgb]{0.53,0.00,0.00}{{#1}}}
    \newcommand{\SpecialCharTok}[1]{\textcolor[rgb]{0.25,0.44,0.63}{{#1}}}
    \newcommand{\VerbatimStringTok}[1]{\textcolor[rgb]{0.25,0.44,0.63}{{#1}}}
    \newcommand{\SpecialStringTok}[1]{\textcolor[rgb]{0.73,0.40,0.53}{{#1}}}
    \newcommand{\ImportTok}[1]{{#1}}
    \newcommand{\DocumentationTok}[1]{\textcolor[rgb]{0.73,0.13,0.13}{\textit{{#1}}}}
    \newcommand{\AnnotationTok}[1]{\textcolor[rgb]{0.38,0.63,0.69}{\textbf{\textit{{#1}}}}}
    \newcommand{\CommentVarTok}[1]{\textcolor[rgb]{0.38,0.63,0.69}{\textbf{\textit{{#1}}}}}
    \newcommand{\VariableTok}[1]{\textcolor[rgb]{0.10,0.09,0.49}{{#1}}}
    \newcommand{\ControlFlowTok}[1]{\textcolor[rgb]{0.00,0.44,0.13}{\textbf{{#1}}}}
    \newcommand{\OperatorTok}[1]{\textcolor[rgb]{0.40,0.40,0.40}{{#1}}}
    \newcommand{\BuiltInTok}[1]{{#1}}
    \newcommand{\ExtensionTok}[1]{{#1}}
    \newcommand{\PreprocessorTok}[1]{\textcolor[rgb]{0.74,0.48,0.00}{{#1}}}
    \newcommand{\AttributeTok}[1]{\textcolor[rgb]{0.49,0.56,0.16}{{#1}}}
    \newcommand{\InformationTok}[1]{\textcolor[rgb]{0.38,0.63,0.69}{\textbf{\textit{{#1}}}}}
    \newcommand{\WarningTok}[1]{\textcolor[rgb]{0.38,0.63,0.69}{\textbf{\textit{{#1}}}}}
    
    
    % Define a nice break command that doesn't care if a line doesn't already
    % exist.
    \def\br{\hspace*{\fill} \\* }
    % Math Jax compatability definitions
    \def\gt{>}
    \def\lt{<}
    % Document parameters
    \title{panek\_tanyuk\_wall\_davieau\_lab1}
    
    
    

    % Pygments definitions
    
\makeatletter
\def\PY@reset{\let\PY@it=\relax \let\PY@bf=\relax%
    \let\PY@ul=\relax \let\PY@tc=\relax%
    \let\PY@bc=\relax \let\PY@ff=\relax}
\def\PY@tok#1{\csname PY@tok@#1\endcsname}
\def\PY@toks#1+{\ifx\relax#1\empty\else%
    \PY@tok{#1}\expandafter\PY@toks\fi}
\def\PY@do#1{\PY@bc{\PY@tc{\PY@ul{%
    \PY@it{\PY@bf{\PY@ff{#1}}}}}}}
\def\PY#1#2{\PY@reset\PY@toks#1+\relax+\PY@do{#2}}

\expandafter\def\csname PY@tok@w\endcsname{\def\PY@tc##1{\textcolor[rgb]{0.73,0.73,0.73}{##1}}}
\expandafter\def\csname PY@tok@c\endcsname{\let\PY@it=\textit\def\PY@tc##1{\textcolor[rgb]{0.25,0.50,0.50}{##1}}}
\expandafter\def\csname PY@tok@cp\endcsname{\def\PY@tc##1{\textcolor[rgb]{0.74,0.48,0.00}{##1}}}
\expandafter\def\csname PY@tok@k\endcsname{\let\PY@bf=\textbf\def\PY@tc##1{\textcolor[rgb]{0.00,0.50,0.00}{##1}}}
\expandafter\def\csname PY@tok@kp\endcsname{\def\PY@tc##1{\textcolor[rgb]{0.00,0.50,0.00}{##1}}}
\expandafter\def\csname PY@tok@kt\endcsname{\def\PY@tc##1{\textcolor[rgb]{0.69,0.00,0.25}{##1}}}
\expandafter\def\csname PY@tok@o\endcsname{\def\PY@tc##1{\textcolor[rgb]{0.40,0.40,0.40}{##1}}}
\expandafter\def\csname PY@tok@ow\endcsname{\let\PY@bf=\textbf\def\PY@tc##1{\textcolor[rgb]{0.67,0.13,1.00}{##1}}}
\expandafter\def\csname PY@tok@nb\endcsname{\def\PY@tc##1{\textcolor[rgb]{0.00,0.50,0.00}{##1}}}
\expandafter\def\csname PY@tok@nf\endcsname{\def\PY@tc##1{\textcolor[rgb]{0.00,0.00,1.00}{##1}}}
\expandafter\def\csname PY@tok@nc\endcsname{\let\PY@bf=\textbf\def\PY@tc##1{\textcolor[rgb]{0.00,0.00,1.00}{##1}}}
\expandafter\def\csname PY@tok@nn\endcsname{\let\PY@bf=\textbf\def\PY@tc##1{\textcolor[rgb]{0.00,0.00,1.00}{##1}}}
\expandafter\def\csname PY@tok@ne\endcsname{\let\PY@bf=\textbf\def\PY@tc##1{\textcolor[rgb]{0.82,0.25,0.23}{##1}}}
\expandafter\def\csname PY@tok@nv\endcsname{\def\PY@tc##1{\textcolor[rgb]{0.10,0.09,0.49}{##1}}}
\expandafter\def\csname PY@tok@no\endcsname{\def\PY@tc##1{\textcolor[rgb]{0.53,0.00,0.00}{##1}}}
\expandafter\def\csname PY@tok@nl\endcsname{\def\PY@tc##1{\textcolor[rgb]{0.63,0.63,0.00}{##1}}}
\expandafter\def\csname PY@tok@ni\endcsname{\let\PY@bf=\textbf\def\PY@tc##1{\textcolor[rgb]{0.60,0.60,0.60}{##1}}}
\expandafter\def\csname PY@tok@na\endcsname{\def\PY@tc##1{\textcolor[rgb]{0.49,0.56,0.16}{##1}}}
\expandafter\def\csname PY@tok@nt\endcsname{\let\PY@bf=\textbf\def\PY@tc##1{\textcolor[rgb]{0.00,0.50,0.00}{##1}}}
\expandafter\def\csname PY@tok@nd\endcsname{\def\PY@tc##1{\textcolor[rgb]{0.67,0.13,1.00}{##1}}}
\expandafter\def\csname PY@tok@s\endcsname{\def\PY@tc##1{\textcolor[rgb]{0.73,0.13,0.13}{##1}}}
\expandafter\def\csname PY@tok@sd\endcsname{\let\PY@it=\textit\def\PY@tc##1{\textcolor[rgb]{0.73,0.13,0.13}{##1}}}
\expandafter\def\csname PY@tok@si\endcsname{\let\PY@bf=\textbf\def\PY@tc##1{\textcolor[rgb]{0.73,0.40,0.53}{##1}}}
\expandafter\def\csname PY@tok@se\endcsname{\let\PY@bf=\textbf\def\PY@tc##1{\textcolor[rgb]{0.73,0.40,0.13}{##1}}}
\expandafter\def\csname PY@tok@sr\endcsname{\def\PY@tc##1{\textcolor[rgb]{0.73,0.40,0.53}{##1}}}
\expandafter\def\csname PY@tok@ss\endcsname{\def\PY@tc##1{\textcolor[rgb]{0.10,0.09,0.49}{##1}}}
\expandafter\def\csname PY@tok@sx\endcsname{\def\PY@tc##1{\textcolor[rgb]{0.00,0.50,0.00}{##1}}}
\expandafter\def\csname PY@tok@m\endcsname{\def\PY@tc##1{\textcolor[rgb]{0.40,0.40,0.40}{##1}}}
\expandafter\def\csname PY@tok@gh\endcsname{\let\PY@bf=\textbf\def\PY@tc##1{\textcolor[rgb]{0.00,0.00,0.50}{##1}}}
\expandafter\def\csname PY@tok@gu\endcsname{\let\PY@bf=\textbf\def\PY@tc##1{\textcolor[rgb]{0.50,0.00,0.50}{##1}}}
\expandafter\def\csname PY@tok@gd\endcsname{\def\PY@tc##1{\textcolor[rgb]{0.63,0.00,0.00}{##1}}}
\expandafter\def\csname PY@tok@gi\endcsname{\def\PY@tc##1{\textcolor[rgb]{0.00,0.63,0.00}{##1}}}
\expandafter\def\csname PY@tok@gr\endcsname{\def\PY@tc##1{\textcolor[rgb]{1.00,0.00,0.00}{##1}}}
\expandafter\def\csname PY@tok@ge\endcsname{\let\PY@it=\textit}
\expandafter\def\csname PY@tok@gs\endcsname{\let\PY@bf=\textbf}
\expandafter\def\csname PY@tok@gp\endcsname{\let\PY@bf=\textbf\def\PY@tc##1{\textcolor[rgb]{0.00,0.00,0.50}{##1}}}
\expandafter\def\csname PY@tok@go\endcsname{\def\PY@tc##1{\textcolor[rgb]{0.53,0.53,0.53}{##1}}}
\expandafter\def\csname PY@tok@gt\endcsname{\def\PY@tc##1{\textcolor[rgb]{0.00,0.27,0.87}{##1}}}
\expandafter\def\csname PY@tok@err\endcsname{\def\PY@bc##1{\setlength{\fboxsep}{0pt}\fcolorbox[rgb]{1.00,0.00,0.00}{1,1,1}{\strut ##1}}}
\expandafter\def\csname PY@tok@kc\endcsname{\let\PY@bf=\textbf\def\PY@tc##1{\textcolor[rgb]{0.00,0.50,0.00}{##1}}}
\expandafter\def\csname PY@tok@kd\endcsname{\let\PY@bf=\textbf\def\PY@tc##1{\textcolor[rgb]{0.00,0.50,0.00}{##1}}}
\expandafter\def\csname PY@tok@kn\endcsname{\let\PY@bf=\textbf\def\PY@tc##1{\textcolor[rgb]{0.00,0.50,0.00}{##1}}}
\expandafter\def\csname PY@tok@kr\endcsname{\let\PY@bf=\textbf\def\PY@tc##1{\textcolor[rgb]{0.00,0.50,0.00}{##1}}}
\expandafter\def\csname PY@tok@bp\endcsname{\def\PY@tc##1{\textcolor[rgb]{0.00,0.50,0.00}{##1}}}
\expandafter\def\csname PY@tok@fm\endcsname{\def\PY@tc##1{\textcolor[rgb]{0.00,0.00,1.00}{##1}}}
\expandafter\def\csname PY@tok@vc\endcsname{\def\PY@tc##1{\textcolor[rgb]{0.10,0.09,0.49}{##1}}}
\expandafter\def\csname PY@tok@vg\endcsname{\def\PY@tc##1{\textcolor[rgb]{0.10,0.09,0.49}{##1}}}
\expandafter\def\csname PY@tok@vi\endcsname{\def\PY@tc##1{\textcolor[rgb]{0.10,0.09,0.49}{##1}}}
\expandafter\def\csname PY@tok@vm\endcsname{\def\PY@tc##1{\textcolor[rgb]{0.10,0.09,0.49}{##1}}}
\expandafter\def\csname PY@tok@sa\endcsname{\def\PY@tc##1{\textcolor[rgb]{0.73,0.13,0.13}{##1}}}
\expandafter\def\csname PY@tok@sb\endcsname{\def\PY@tc##1{\textcolor[rgb]{0.73,0.13,0.13}{##1}}}
\expandafter\def\csname PY@tok@sc\endcsname{\def\PY@tc##1{\textcolor[rgb]{0.73,0.13,0.13}{##1}}}
\expandafter\def\csname PY@tok@dl\endcsname{\def\PY@tc##1{\textcolor[rgb]{0.73,0.13,0.13}{##1}}}
\expandafter\def\csname PY@tok@s2\endcsname{\def\PY@tc##1{\textcolor[rgb]{0.73,0.13,0.13}{##1}}}
\expandafter\def\csname PY@tok@sh\endcsname{\def\PY@tc##1{\textcolor[rgb]{0.73,0.13,0.13}{##1}}}
\expandafter\def\csname PY@tok@s1\endcsname{\def\PY@tc##1{\textcolor[rgb]{0.73,0.13,0.13}{##1}}}
\expandafter\def\csname PY@tok@mb\endcsname{\def\PY@tc##1{\textcolor[rgb]{0.40,0.40,0.40}{##1}}}
\expandafter\def\csname PY@tok@mf\endcsname{\def\PY@tc##1{\textcolor[rgb]{0.40,0.40,0.40}{##1}}}
\expandafter\def\csname PY@tok@mh\endcsname{\def\PY@tc##1{\textcolor[rgb]{0.40,0.40,0.40}{##1}}}
\expandafter\def\csname PY@tok@mi\endcsname{\def\PY@tc##1{\textcolor[rgb]{0.40,0.40,0.40}{##1}}}
\expandafter\def\csname PY@tok@il\endcsname{\def\PY@tc##1{\textcolor[rgb]{0.40,0.40,0.40}{##1}}}
\expandafter\def\csname PY@tok@mo\endcsname{\def\PY@tc##1{\textcolor[rgb]{0.40,0.40,0.40}{##1}}}
\expandafter\def\csname PY@tok@ch\endcsname{\let\PY@it=\textit\def\PY@tc##1{\textcolor[rgb]{0.25,0.50,0.50}{##1}}}
\expandafter\def\csname PY@tok@cm\endcsname{\let\PY@it=\textit\def\PY@tc##1{\textcolor[rgb]{0.25,0.50,0.50}{##1}}}
\expandafter\def\csname PY@tok@cpf\endcsname{\let\PY@it=\textit\def\PY@tc##1{\textcolor[rgb]{0.25,0.50,0.50}{##1}}}
\expandafter\def\csname PY@tok@c1\endcsname{\let\PY@it=\textit\def\PY@tc##1{\textcolor[rgb]{0.25,0.50,0.50}{##1}}}
\expandafter\def\csname PY@tok@cs\endcsname{\let\PY@it=\textit\def\PY@tc##1{\textcolor[rgb]{0.25,0.50,0.50}{##1}}}

\def\PYZbs{\char`\\}
\def\PYZus{\char`\_}
\def\PYZob{\char`\{}
\def\PYZcb{\char`\}}
\def\PYZca{\char`\^}
\def\PYZam{\char`\&}
\def\PYZlt{\char`\<}
\def\PYZgt{\char`\>}
\def\PYZsh{\char`\#}
\def\PYZpc{\char`\%}
\def\PYZdl{\char`\$}
\def\PYZhy{\char`\-}
\def\PYZsq{\char`\'}
\def\PYZdq{\char`\"}
\def\PYZti{\char`\~}
% for compatibility with earlier versions
\def\PYZat{@}
\def\PYZlb{[}
\def\PYZrb{]}
\makeatother


    % Exact colors from NB
    \definecolor{incolor}{rgb}{0.0, 0.0, 0.5}
    \definecolor{outcolor}{rgb}{0.545, 0.0, 0.0}



    
    % Prevent overflowing lines due to hard-to-break entities
    \sloppy 
    % Setup hyperref package
    \hypersetup{
      breaklinks=true,  % so long urls are correctly broken across lines
      colorlinks=true,
      urlcolor=urlcolor,
      linkcolor=linkcolor,
      citecolor=citecolor,
      }
    % Slightly bigger margins than the latex defaults
    
    \geometry{verbose,tmargin=1in,bmargin=1in,lmargin=1in,rmargin=1in}
    
    

    \begin{document}
    
    
    \maketitle
    
    

    
    \section{First Project Work Week
Assignment}\label{first-project-work-week-assignment}

    \subsection{Business Understanding}\label{business-understanding}

Business applications of predicting an overall default rate could
include capcacity planning and financial reporting. However, the ability
to predict whether an individual customer will default is also an
important aspect of managing the profitability of a credit card
business.

Projecting the likelihood of default for a given customer can be used
to:

\begin{enumerate}
\def\labelenumi{\arabic{enumi}.}
\tightlist
\item
  Determine the collection strategy if a customer misses a payment
\item
  Determine the appropriate credit limit for that customer
\end{enumerate}

Use of inaccurate predictions to determine how a customer will be
treated can adversely impact profitability. - If the risk is
overestimated, collection efforts may be too intense thereby alienating
customers and unnecessarily constraining credit lines. Restricing credit
lines too much may inhibit customers' ability and willingness to use the
product. - If the risk is underestimated, the bank will incur higher
losses than it might otherwise. We expect that a more acceptable bias
would be to overestimate the likelihood of default rather than
underestimate it.

\paragraph{** Goal ** The goal of this analysis is to identify basic
patterns in the
data.}\label{goal-the-goal-of-this-analysis-is-to-identify-basic-patterns-in-the-data.}

In subsequent studies we will predict the probablity of default for
credit card customers. We will set aside a portion of the data for
validation (the test set), and use Logistic Regression on the remaining
training set.

The effectiveness of the model in predicting an overall default rate
will be measured by its performance when applied to the test data set.
If the actual results are within 10\% of the estimate, we will deem it
to be successful. We will also use the AUC against the test set.
Judgmentally, we will consider an AUC of 80\% or more acceptable.

To test the effectiveness of the model for use in determining the course
of action with respect to a specific customer we will look for
specificity and sensitivity rates at certain probabilities of default.
In order to determine whether a "lighter" collection strategy should be
used, we will look for those probabilities where the sensitivity (true
positive rate) is greater than 90\%. In order to determine that a
request to increase credit will be declined, we will look for those
probabilities where the specificity is 90\% or more.

    \subsection{Data Meaning Type}\label{data-meaning-type}

\paragraph{Attribute Information}\label{attribute-information}

The data used is "Default of Credit Card Clients" from UCI. It was
attained by I-Cheng Yeh with Chung Hua University and Tamkang University
in Taiwan. The original goal was to predict default rates.

The data has a 6 month history of 30,000 Taiwanese credit account
balances and transactions. Each observation contains a binary reponse
variable "default" with values 1 indicating a default occured and 0
indicating no default occured.

The following explanatory variables are included:

\begin{itemize}
\item
  LIMIT\_BAL = Total credit amount allowed
\item
  SEX

  \begin{itemize}
  \tightlist
  \item
    1 = Male
  \item
    2 = Female
  \end{itemize}
\item
  EDUCATION

  \begin{itemize}
  \tightlist
  \item
    1 = Graduate School
  \item
    2 = University
  \item
    3 = High School
  \item
    4 = Other
  \end{itemize}
\item
  MARRIAGE

  \begin{itemize}
  \tightlist
  \item
    1 = Married
  \item
    2 = Single
  \item
    3 = Other
  \end{itemize}
\item
  AGE = Credit holder age in years
\end{itemize}

Payment history (2005) - PAY\_0 = September - PAY\_2 = August - PAY\_3 =
July - PAY\_4 = June - PAY\_5 = May - PAY\_6 = April - -1 = payment
received on time - 1 = payment received one month late - 2 = payment
received two months late - "......" - 9 = payment received nine months
late or more

Statement amount (NT dollars, 2005) - BILL\_AMT1 = September -
BILL\_AMT2 = August - BILL\_AMT3 = July - BILL\_AMT4 = June - BILL\_AMT5
= May - BILL\_AMT6 = April

Payment amount (NT dollars, 2005). - PAY\_AMT1 = September - PAY\_AMT2 =
August - PAY\_AMT3 = July - PAY\_AMT4 = June - PAY\_AMT5 = May -
PAY\_AMT6 = April

Original Source Data Set Information\\
https://archive.ics.uci.edu/ml/datasets/default+of+credit+card+clients\#

    \begin{Verbatim}[commandchars=\\\{\}]
{\color{incolor}In [{\color{incolor}1}]:} \PY{c+c1}{\PYZsh{}import libraries}
        \PY{k+kn}{import} \PY{n+nn}{pandas} \PY{k}{as} \PY{n+nn}{pd}
        \PY{k+kn}{import} \PY{n+nn}{seaborn} \PY{k}{as} \PY{n+nn}{sns}
        \PY{k+kn}{import} \PY{n+nn}{numpy} \PY{k}{as} \PY{n+nn}{np}
        \PY{k+kn}{import} \PY{n+nn}{matplotlib}\PY{n+nn}{.}\PY{n+nn}{pyplot} \PY{k}{as} \PY{n+nn}{plt}
        \PY{k+kn}{import} \PY{n+nn}{warnings}
        \PY{n}{warnings}\PY{o}{.}\PY{n}{simplefilter}\PY{p}{(}\PY{l+s+s1}{\PYZsq{}}\PY{l+s+s1}{ignore}\PY{l+s+s1}{\PYZsq{}}\PY{p}{,} \PY{n+ne}{DeprecationWarning}\PY{p}{)}
        
        \PY{c+c1}{\PYZsh{}import the data}
        \PY{n}{df} \PY{o}{=} \PY{n}{pd}\PY{o}{.}\PY{n}{read\PYZus{}csv}\PY{p}{(}\PY{l+s+s1}{\PYZsq{}}\PY{l+s+s1}{Input/DefaultCreditcardClients.csv}\PY{l+s+s1}{\PYZsq{}}\PY{p}{)}
        \PY{n}{df}\PY{o}{.}\PY{n}{rename}\PY{p}{(}\PY{n}{columns}\PY{o}{=}\PY{p}{\PYZob{}}\PY{l+s+s1}{\PYZsq{}}\PY{l+s+s1}{default payment next month}\PY{l+s+s1}{\PYZsq{}}\PY{p}{:}\PY{l+s+s1}{\PYZsq{}}\PY{l+s+s1}{default}\PY{l+s+s1}{\PYZsq{}}\PY{p}{\PYZcb{}}\PY{p}{,} \PY{n}{inplace}\PY{o}{=}\PY{k+kc}{True}\PY{p}{)}
        \PY{n}{df}\PY{o}{.}\PY{n}{index} \PY{o}{=} \PY{n}{df}\PY{o}{.}\PY{n}{ID}
        \PY{k}{if} \PY{l+s+s1}{\PYZsq{}}\PY{l+s+s1}{ID}\PY{l+s+s1}{\PYZsq{}} \PY{o+ow}{in} \PY{n}{df}\PY{p}{:}
            \PY{k}{del} \PY{n}{df}\PY{p}{[}\PY{l+s+s1}{\PYZsq{}}\PY{l+s+s1}{ID}\PY{l+s+s1}{\PYZsq{}}\PY{p}{]}
        \PY{n}{df}\PY{o}{.}\PY{n}{head}\PY{p}{(}\PY{p}{)}
\end{Verbatim}


\begin{Verbatim}[commandchars=\\\{\}]
{\color{outcolor}Out[{\color{outcolor}1}]:}     LIMIT\_BAL  SEX  EDUCATION  MARRIAGE  AGE  PAY\_0  PAY\_2  PAY\_3  PAY\_4  \textbackslash{}
        ID                                                                         
        1       20000    2          2         1   24      2      2     -1     -1   
        2      120000    2          2         2   26     -1      2      0      0   
        3       90000    2          2         2   34      0      0      0      0   
        4       50000    2          2         1   37      0      0      0      0   
        5       50000    1          2         1   57     -1      0     -1      0   
        
            PAY\_5   {\ldots}     BILL\_AMT4  BILL\_AMT5  BILL\_AMT6  PAY\_AMT1  PAY\_AMT2  \textbackslash{}
        ID          {\ldots}                                                           
        1      -2   {\ldots}             0          0          0         0       689   
        2       0   {\ldots}          3272       3455       3261         0      1000   
        3       0   {\ldots}         14331      14948      15549      1518      1500   
        4       0   {\ldots}         28314      28959      29547      2000      2019   
        5       0   {\ldots}         20940      19146      19131      2000     36681   
        
            PAY\_AMT3  PAY\_AMT4  PAY\_AMT5  PAY\_AMT6  default  
        ID                                                   
        1          0         0         0         0        1  
        2       1000      1000         0      2000        1  
        3       1000      1000      1000      5000        0  
        4       1200      1100      1069      1000        0  
        5      10000      9000       689       679        0  
        
        [5 rows x 24 columns]
\end{Verbatim}
            
    The table above shows a peek at the first 5 records in the dataset.

    \begin{Verbatim}[commandchars=\\\{\}]
{\color{incolor}In [{\color{incolor}2}]:} \PY{n}{df}\PY{o}{.}\PY{n}{dtypes}
\end{Verbatim}


\begin{Verbatim}[commandchars=\\\{\}]
{\color{outcolor}Out[{\color{outcolor}2}]:} LIMIT\_BAL    int64
        SEX          int64
        EDUCATION    int64
        MARRIAGE     int64
        AGE          int64
        PAY\_0        int64
        PAY\_2        int64
        PAY\_3        int64
        PAY\_4        int64
        PAY\_5        int64
        PAY\_6        int64
        BILL\_AMT1    int64
        BILL\_AMT2    int64
        BILL\_AMT3    int64
        BILL\_AMT4    int64
        BILL\_AMT5    int64
        BILL\_AMT6    int64
        PAY\_AMT1     int64
        PAY\_AMT2     int64
        PAY\_AMT3     int64
        PAY\_AMT4     int64
        PAY\_AMT5     int64
        PAY\_AMT6     int64
        default      int64
        dtype: object
\end{Verbatim}
            
    Pandas defaulted all data types to integer. The source has no explicit
data type descriptions but there is enough context to safely change the
datatypes of all continuous variables to floats.

    \begin{Verbatim}[commandchars=\\\{\}]
{\color{incolor}In [{\color{incolor}3}]:} \PY{c+c1}{\PYZsh{}Create Lists for Analysis}
        \PY{n}{BillsAndPayments}\PY{o}{=}\PY{p}{[}\PY{l+s+s1}{\PYZsq{}}\PY{l+s+s1}{BILL\PYZus{}AMT1}\PY{l+s+s1}{\PYZsq{}}\PY{p}{,}\PY{l+s+s1}{\PYZsq{}}\PY{l+s+s1}{BILL\PYZus{}AMT2}\PY{l+s+s1}{\PYZsq{}}\PY{p}{,}\PY{l+s+s1}{\PYZsq{}}\PY{l+s+s1}{BILL\PYZus{}AMT3}\PY{l+s+s1}{\PYZsq{}}\PY{p}{,}\PY{l+s+s1}{\PYZsq{}}\PY{l+s+s1}{BILL\PYZus{}AMT4}\PY{l+s+s1}{\PYZsq{}}\PY{p}{,}\PY{l+s+s1}{\PYZsq{}}\PY{l+s+s1}{BILL\PYZus{}AMT5}\PY{l+s+s1}{\PYZsq{}}\PY{p}{,}\PY{l+s+s1}{\PYZsq{}}\PY{l+s+s1}{BILL\PYZus{}AMT6}\PY{l+s+s1}{\PYZsq{}}\PY{p}{,}\PY{l+s+s1}{\PYZsq{}}\PY{l+s+s1}{PAY\PYZus{}AMT1}\PY{l+s+s1}{\PYZsq{}}\PY{p}{,}\PY{l+s+s1}{\PYZsq{}}\PY{l+s+s1}{PAY\PYZus{}AMT2}\PY{l+s+s1}{\PYZsq{}}\PY{p}{,}\PY{l+s+s1}{\PYZsq{}}\PY{l+s+s1}{PAY\PYZus{}AMT3}\PY{l+s+s1}{\PYZsq{}}\PY{p}{,}\PY{l+s+s1}{\PYZsq{}}\PY{l+s+s1}{PAY\PYZus{}AMT4}\PY{l+s+s1}{\PYZsq{}}\PY{p}{,}\PY{l+s+s1}{\PYZsq{}}\PY{l+s+s1}{PAY\PYZus{}AMT5}\PY{l+s+s1}{\PYZsq{}}\PY{p}{,}\PY{l+s+s1}{\PYZsq{}}\PY{l+s+s1}{PAY\PYZus{}AMT6}\PY{l+s+s1}{\PYZsq{}}\PY{p}{,}\PY{l+s+s1}{\PYZsq{}}\PY{l+s+s1}{default}\PY{l+s+s1}{\PYZsq{}}\PY{p}{]}
        \PY{n}{Bills}\PY{o}{=}\PY{p}{[}\PY{l+s+s1}{\PYZsq{}}\PY{l+s+s1}{LIMIT\PYZus{}BAL}\PY{l+s+s1}{\PYZsq{}}\PY{p}{,}\PY{l+s+s1}{\PYZsq{}}\PY{l+s+s1}{BILL\PYZus{}AMT1}\PY{l+s+s1}{\PYZsq{}}\PY{p}{,}\PY{l+s+s1}{\PYZsq{}}\PY{l+s+s1}{BILL\PYZus{}AMT2}\PY{l+s+s1}{\PYZsq{}}\PY{p}{,}\PY{l+s+s1}{\PYZsq{}}\PY{l+s+s1}{BILL\PYZus{}AMT3}\PY{l+s+s1}{\PYZsq{}}\PY{p}{,}\PY{l+s+s1}{\PYZsq{}}\PY{l+s+s1}{BILL\PYZus{}AMT4}\PY{l+s+s1}{\PYZsq{}}\PY{p}{,}\PY{l+s+s1}{\PYZsq{}}\PY{l+s+s1}{BILL\PYZus{}AMT5}\PY{l+s+s1}{\PYZsq{}}\PY{p}{,}\PY{l+s+s1}{\PYZsq{}}\PY{l+s+s1}{BILL\PYZus{}AMT6}\PY{l+s+s1}{\PYZsq{}}\PY{p}{,}\PY{l+s+s1}{\PYZsq{}}\PY{l+s+s1}{default}\PY{l+s+s1}{\PYZsq{}}\PY{p}{]}
        \PY{n}{Payments}\PY{o}{=}\PY{p}{[}\PY{l+s+s1}{\PYZsq{}}\PY{l+s+s1}{PAY\PYZus{}AMT1}\PY{l+s+s1}{\PYZsq{}}\PY{p}{,}\PY{l+s+s1}{\PYZsq{}}\PY{l+s+s1}{PAY\PYZus{}AMT2}\PY{l+s+s1}{\PYZsq{}}\PY{p}{,}\PY{l+s+s1}{\PYZsq{}}\PY{l+s+s1}{PAY\PYZus{}AMT3}\PY{l+s+s1}{\PYZsq{}}\PY{p}{,}\PY{l+s+s1}{\PYZsq{}}\PY{l+s+s1}{PAY\PYZus{}AMT4}\PY{l+s+s1}{\PYZsq{}}\PY{p}{,}\PY{l+s+s1}{\PYZsq{}}\PY{l+s+s1}{PAY\PYZus{}AMT5}\PY{l+s+s1}{\PYZsq{}}\PY{p}{,}\PY{l+s+s1}{\PYZsq{}}\PY{l+s+s1}{PAY\PYZus{}AMT6}\PY{l+s+s1}{\PYZsq{}}\PY{p}{,}\PY{l+s+s1}{\PYZsq{}}\PY{l+s+s1}{default}\PY{l+s+s1}{\PYZsq{}}\PY{p}{]}
        \PY{n}{PayStatus}\PY{o}{=}\PY{p}{[}\PY{l+s+s1}{\PYZsq{}}\PY{l+s+s1}{PAY\PYZus{}0}\PY{l+s+s1}{\PYZsq{}}\PY{p}{,}\PY{l+s+s1}{\PYZsq{}}\PY{l+s+s1}{PAY\PYZus{}2}\PY{l+s+s1}{\PYZsq{}}\PY{p}{,}\PY{l+s+s1}{\PYZsq{}}\PY{l+s+s1}{PAY\PYZus{}3}\PY{l+s+s1}{\PYZsq{}}\PY{p}{,}\PY{l+s+s1}{\PYZsq{}}\PY{l+s+s1}{PAY\PYZus{}4}\PY{l+s+s1}{\PYZsq{}}\PY{p}{,}\PY{l+s+s1}{\PYZsq{}}\PY{l+s+s1}{PAY\PYZus{}5}\PY{l+s+s1}{\PYZsq{}}\PY{p}{,}\PY{l+s+s1}{\PYZsq{}}\PY{l+s+s1}{PAY\PYZus{}6}\PY{l+s+s1}{\PYZsq{}}\PY{p}{,}\PY{l+s+s1}{\PYZsq{}}\PY{l+s+s1}{default}\PY{l+s+s1}{\PYZsq{}}\PY{p}{]}
        \PY{n}{PayStausOnly} \PY{o}{=} \PY{p}{[}\PY{l+s+s1}{\PYZsq{}}\PY{l+s+s1}{PAY\PYZus{}0}\PY{l+s+s1}{\PYZsq{}}\PY{p}{,}\PY{l+s+s1}{\PYZsq{}}\PY{l+s+s1}{PAY\PYZus{}2}\PY{l+s+s1}{\PYZsq{}}\PY{p}{,}\PY{l+s+s1}{\PYZsq{}}\PY{l+s+s1}{PAY\PYZus{}3}\PY{l+s+s1}{\PYZsq{}}\PY{p}{,}\PY{l+s+s1}{\PYZsq{}}\PY{l+s+s1}{PAY\PYZus{}4}\PY{l+s+s1}{\PYZsq{}}\PY{p}{,}\PY{l+s+s1}{\PYZsq{}}\PY{l+s+s1}{PAY\PYZus{}5}\PY{l+s+s1}{\PYZsq{}}\PY{p}{,}\PY{l+s+s1}{\PYZsq{}}\PY{l+s+s1}{PAY\PYZus{}6}\PY{l+s+s1}{\PYZsq{}}\PY{p}{]}
        \PY{n}{continuous\PYZus{}features} \PY{o}{=} \PY{p}{[}\PY{l+s+s1}{\PYZsq{}}\PY{l+s+s1}{LIMIT\PYZus{}BAL}\PY{l+s+s1}{\PYZsq{}}\PY{p}{,} \PY{l+s+s1}{\PYZsq{}}\PY{l+s+s1}{BILL\PYZus{}AMT1}\PY{l+s+s1}{\PYZsq{}}\PY{p}{,} \PY{l+s+s1}{\PYZsq{}}\PY{l+s+s1}{BILL\PYZus{}AMT2}\PY{l+s+s1}{\PYZsq{}}\PY{p}{,}\PY{l+s+s1}{\PYZsq{}}\PY{l+s+s1}{BILL\PYZus{}AMT3}\PY{l+s+s1}{\PYZsq{}}\PY{p}{,}
                               \PY{l+s+s1}{\PYZsq{}}\PY{l+s+s1}{BILL\PYZus{}AMT4}\PY{l+s+s1}{\PYZsq{}}\PY{p}{,} \PY{l+s+s1}{\PYZsq{}}\PY{l+s+s1}{BILL\PYZus{}AMT5}\PY{l+s+s1}{\PYZsq{}}\PY{p}{,} \PY{l+s+s1}{\PYZsq{}}\PY{l+s+s1}{BILL\PYZus{}AMT6}\PY{l+s+s1}{\PYZsq{}}\PY{p}{,} \PY{l+s+s1}{\PYZsq{}}\PY{l+s+s1}{PAY\PYZus{}AMT1}\PY{l+s+s1}{\PYZsq{}}\PY{p}{,}
                               \PY{l+s+s1}{\PYZsq{}}\PY{l+s+s1}{PAY\PYZus{}AMT2}\PY{l+s+s1}{\PYZsq{}}\PY{p}{,} \PY{l+s+s1}{\PYZsq{}}\PY{l+s+s1}{PAY\PYZus{}AMT3}\PY{l+s+s1}{\PYZsq{}}\PY{p}{,} \PY{l+s+s1}{\PYZsq{}}\PY{l+s+s1}{PAY\PYZus{}AMT4}\PY{l+s+s1}{\PYZsq{}}\PY{p}{,} \PY{l+s+s1}{\PYZsq{}}\PY{l+s+s1}{PAY\PYZus{}AMT5}\PY{l+s+s1}{\PYZsq{}}\PY{p}{,}
                               \PY{l+s+s1}{\PYZsq{}}\PY{l+s+s1}{PAY\PYZus{}AMT6}\PY{l+s+s1}{\PYZsq{}}\PY{p}{]}
        \PY{n}{ordinal\PYZus{}features} \PY{o}{=} \PY{p}{[}\PY{l+s+s1}{\PYZsq{}}\PY{l+s+s1}{EDUCATION}\PY{l+s+s1}{\PYZsq{}}\PY{p}{,} \PY{l+s+s1}{\PYZsq{}}\PY{l+s+s1}{MARRIAGE}\PY{l+s+s1}{\PYZsq{}}\PY{p}{,} \PY{l+s+s1}{\PYZsq{}}\PY{l+s+s1}{AGE}\PY{l+s+s1}{\PYZsq{}}\PY{p}{,} \PY{l+s+s1}{\PYZsq{}}\PY{l+s+s1}{PAY\PYZus{}0}\PY{l+s+s1}{\PYZsq{}}\PY{p}{,}\PY{l+s+s1}{\PYZsq{}}\PY{l+s+s1}{PAY\PYZus{}2}\PY{l+s+s1}{\PYZsq{}}\PY{p}{,} \PY{l+s+s1}{\PYZsq{}}\PY{l+s+s1}{PAY\PYZus{}3}\PY{l+s+s1}{\PYZsq{}}\PY{p}{,} \PY{l+s+s1}{\PYZsq{}}\PY{l+s+s1}{PAY\PYZus{}4}\PY{l+s+s1}{\PYZsq{}}\PY{p}{,} \PY{l+s+s1}{\PYZsq{}}\PY{l+s+s1}{PAY\PYZus{}5}\PY{l+s+s1}{\PYZsq{}}\PY{p}{,} \PY{l+s+s1}{\PYZsq{}}\PY{l+s+s1}{PAY\PYZus{}6}\PY{l+s+s1}{\PYZsq{}}\PY{p}{,}\PY{l+s+s1}{\PYZsq{}}\PY{l+s+s1}{default}\PY{l+s+s1}{\PYZsq{}}\PY{p}{]}
        \PY{n}{pca\PYZus{}features} \PY{o}{=} \PY{p}{[}\PY{l+s+s1}{\PYZsq{}}\PY{l+s+s1}{LIMIT\PYZus{}BAL}\PY{l+s+s1}{\PYZsq{}}\PY{p}{,}\PY{l+s+s1}{\PYZsq{}}\PY{l+s+s1}{BILL\PYZus{}AMT1}\PY{l+s+s1}{\PYZsq{}}\PY{p}{,} \PY{l+s+s1}{\PYZsq{}}\PY{l+s+s1}{BILL\PYZus{}AMT2}\PY{l+s+s1}{\PYZsq{}}\PY{p}{,}\PY{l+s+s1}{\PYZsq{}}\PY{l+s+s1}{BILL\PYZus{}AMT3}\PY{l+s+s1}{\PYZsq{}}\PY{p}{,}\PY{l+s+s1}{\PYZsq{}}\PY{l+s+s1}{BILL\PYZus{}AMT4}\PY{l+s+s1}{\PYZsq{}}\PY{p}{,} \PY{l+s+s1}{\PYZsq{}}\PY{l+s+s1}{BILL\PYZus{}AMT5}\PY{l+s+s1}{\PYZsq{}}\PY{p}{,} \PY{l+s+s1}{\PYZsq{}}\PY{l+s+s1}{BILL\PYZus{}AMT6}\PY{l+s+s1}{\PYZsq{}}\PY{p}{,}
                        \PY{l+s+s1}{\PYZsq{}}\PY{l+s+s1}{PAY\PYZus{}AMT1}\PY{l+s+s1}{\PYZsq{}}\PY{p}{,}\PY{l+s+s1}{\PYZsq{}}\PY{l+s+s1}{PAY\PYZus{}AMT2}\PY{l+s+s1}{\PYZsq{}}\PY{p}{,} \PY{l+s+s1}{\PYZsq{}}\PY{l+s+s1}{PAY\PYZus{}AMT3}\PY{l+s+s1}{\PYZsq{}}\PY{p}{,} \PY{l+s+s1}{\PYZsq{}}\PY{l+s+s1}{PAY\PYZus{}AMT4}\PY{l+s+s1}{\PYZsq{}}\PY{p}{,} \PY{l+s+s1}{\PYZsq{}}\PY{l+s+s1}{PAY\PYZus{}AMT5}\PY{l+s+s1}{\PYZsq{}}\PY{p}{,}\PY{l+s+s1}{\PYZsq{}}\PY{l+s+s1}{PAY\PYZus{}AMT6}\PY{l+s+s1}{\PYZsq{}}\PY{p}{]}
        \PY{c+c1}{\PYZsh{}Convert datatypes}
        \PY{n}{df}\PY{p}{[}\PY{n}{continuous\PYZus{}features}\PY{p}{]} \PY{o}{=} \PY{n}{df}\PY{p}{[}\PY{n}{continuous\PYZus{}features}\PY{p}{]}\PY{o}{.}\PY{n}{astype}\PY{p}{(}\PY{n}{np}\PY{o}{.}\PY{n}{float64}\PY{p}{)}
        \PY{n}{df}\PY{p}{[}\PY{n}{ordinal\PYZus{}features}\PY{p}{]} \PY{o}{=} \PY{n}{df}\PY{p}{[}\PY{n}{ordinal\PYZus{}features}\PY{p}{]}\PY{o}{.}\PY{n}{astype}\PY{p}{(}\PY{n}{np}\PY{o}{.}\PY{n}{int64}\PY{p}{)}
        \PY{n}{df\PYZus{}pca} \PY{o}{=} \PY{n}{df}\PY{p}{[}\PY{n}{pca\PYZus{}features}\PY{p}{]}\PY{o}{.}\PY{n}{astype}\PY{p}{(}\PY{n}{np}\PY{o}{.}\PY{n}{int64}\PY{p}{)}
        \PY{n}{df}\PY{o}{.}\PY{n}{dtypes}
\end{Verbatim}


\begin{Verbatim}[commandchars=\\\{\}]
{\color{outcolor}Out[{\color{outcolor}3}]:} LIMIT\_BAL    float64
        SEX            int64
        EDUCATION      int64
        MARRIAGE       int64
        AGE            int64
        PAY\_0          int64
        PAY\_2          int64
        PAY\_3          int64
        PAY\_4          int64
        PAY\_5          int64
        PAY\_6          int64
        BILL\_AMT1    float64
        BILL\_AMT2    float64
        BILL\_AMT3    float64
        BILL\_AMT4    float64
        BILL\_AMT5    float64
        BILL\_AMT6    float64
        PAY\_AMT1     float64
        PAY\_AMT2     float64
        PAY\_AMT3     float64
        PAY\_AMT4     float64
        PAY\_AMT5     float64
        PAY\_AMT6     float64
        default        int64
        dtype: object
\end{Verbatim}
            
    \subsection{Data Quality}\label{data-quality}

There were no missing values identified. However in cases where the bill
amount is \$0 and the payment amount is \$0 presented challenges which
are addressed only when necessary in each portion of analysis.

\paragraph{Field definitions and
naming:}\label{field-definitions-and-naming}

\begin{enumerate}
\def\labelenumi{\arabic{enumi}.}
\tightlist
\item
  The payment fields range from PAY\_0 to PAY\_6. There is no PAY\_1
  named field. This raises suspicion; we will assume that there is no
  missing column and proceed with caution.
\item
  The descriptions of values -2 and 0 in the pay fields are not
  provided. Based on visual inspection, a value of -2 appears to
  indicate that no payment is due because the account has a credit
  balance. Because a value of -1 means "Paid on time" and a value of 1
  means one month late, we are assuming that a value of zero means less
  than one month late. It may be that these are missing values coded as
  0. About half of the data set has a value of 0 for these attributes.
\item
  The value 0 is found in the marriage attribute but is not defined.
  There are only 54 instances. We added these to the "Other" category.
\end{enumerate}

\paragraph{Suspicious values:}\label{suspicious-values}

\begin{enumerate}
\def\labelenumi{\arabic{enumi}.}
\tightlist
\item
  There are only 34 instances where the payment status had a value of 1
  for two months in a row. This is highly unusual given the propensity
  for customers with status values of 1 or 2 to remain in their current
  payment status for one month.
\item
  It appears that the first month the value of 1 is used is in PAY\_0
  with only a few observations prior to that. Therfore it seems that
  there is a methodology change part-way through the data series. We
  will proceed with caution since the number of observations is small.
\item
  There are nonsensical instances where the Payment Status implies late
  payment but the amount billed was zero or less. There are 1,769 such
  cases out of 150,000 possible payments.
\end{enumerate}

\paragraph{Outliers:}\label{outliers}

The range of values in the Payment, Amount Billed and Credit Limit
fields are extremely wide and right-skewed. We explored a sample of
outliers and concluded that the observations were legitimate. High
payments were consistent with amounts billed and high balances were
often recurring which is consistent with the credit limits.

The distributions of other variables seem reasonable.

    \subsection{Simple Statistics}\label{simple-statistics}

Visualize appropriate statistics (e.g., range, mode, mean, median,
variance, counts) for a subset of attributes. Describe anything
meaningful you found from this or if you found something potentially
interesting. Note: You can also use data from other sources for
comparison. Explain why the statistics run are meaningful.

    \begin{Verbatim}[commandchars=\\\{\}]
{\color{incolor}In [{\color{incolor}4}]:} \PY{c+c1}{\PYZsh{} Describing the data set in two sections so we can see the values for all columnts.}
        \PY{n}{df}\PY{p}{[}\PY{p}{[}\PY{l+s+s1}{\PYZsq{}}\PY{l+s+s1}{EDUCATION}\PY{l+s+s1}{\PYZsq{}}\PY{p}{,} \PY{l+s+s1}{\PYZsq{}}\PY{l+s+s1}{MARRIAGE}\PY{l+s+s1}{\PYZsq{}}\PY{p}{,} \PY{l+s+s1}{\PYZsq{}}\PY{l+s+s1}{AGE}\PY{l+s+s1}{\PYZsq{}}\PY{p}{,} \PY{l+s+s1}{\PYZsq{}}\PY{l+s+s1}{PAY\PYZus{}0}\PY{l+s+s1}{\PYZsq{}}\PY{p}{,}\PY{l+s+s1}{\PYZsq{}}\PY{l+s+s1}{PAY\PYZus{}2}\PY{l+s+s1}{\PYZsq{}}\PY{p}{,} \PY{l+s+s1}{\PYZsq{}}\PY{l+s+s1}{PAY\PYZus{}3}\PY{l+s+s1}{\PYZsq{}}\PY{p}{,} \PY{l+s+s1}{\PYZsq{}}\PY{l+s+s1}{PAY\PYZus{}4}\PY{l+s+s1}{\PYZsq{}}\PY{p}{,} \PY{l+s+s1}{\PYZsq{}}\PY{l+s+s1}{PAY\PYZus{}5}\PY{l+s+s1}{\PYZsq{}}\PY{p}{,} \PY{l+s+s1}{\PYZsq{}}\PY{l+s+s1}{PAY\PYZus{}6}\PY{l+s+s1}{\PYZsq{}}\PY{p}{,}\PY{l+s+s1}{\PYZsq{}}\PY{l+s+s1}{default}\PY{l+s+s1}{\PYZsq{}}\PY{p}{]}\PY{p}{]}\PY{o}{.}\PY{n}{describe}\PY{p}{(}\PY{p}{)}
\end{Verbatim}


\begin{Verbatim}[commandchars=\\\{\}]
{\color{outcolor}Out[{\color{outcolor}4}]:}           EDUCATION      MARRIAGE           AGE         PAY\_0         PAY\_2  \textbackslash{}
        count  30000.000000  30000.000000  30000.000000  30000.000000  30000.000000   
        mean       1.853133      1.551867     35.485500     -0.016700     -0.133767   
        std        0.790349      0.521970      9.217904      1.123802      1.197186   
        min        0.000000      0.000000     21.000000     -2.000000     -2.000000   
        25\%        1.000000      1.000000     28.000000     -1.000000     -1.000000   
        50\%        2.000000      2.000000     34.000000      0.000000      0.000000   
        75\%        2.000000      2.000000     41.000000      0.000000      0.000000   
        max        6.000000      3.000000     79.000000      8.000000      8.000000   
        
                      PAY\_3         PAY\_4         PAY\_5         PAY\_6       default  
        count  30000.000000  30000.000000  30000.000000  30000.000000  30000.000000  
        mean      -0.166200     -0.220667     -0.266200     -0.291100      0.221200  
        std        1.196868      1.169139      1.133187      1.149988      0.415062  
        min       -2.000000     -2.000000     -2.000000     -2.000000      0.000000  
        25\%       -1.000000     -1.000000     -1.000000     -1.000000      0.000000  
        50\%        0.000000      0.000000      0.000000      0.000000      0.000000  
        75\%        0.000000      0.000000      0.000000      0.000000      0.000000  
        max        8.000000      8.000000      8.000000      8.000000      1.000000  
\end{Verbatim}
            
    \begin{Verbatim}[commandchars=\\\{\}]
{\color{incolor}In [{\color{incolor}5}]:} \PY{n}{df}\PY{p}{[}\PY{n}{continuous\PYZus{}features}\PY{p}{]}\PY{o}{.}\PY{n}{describe}\PY{p}{(}\PY{p}{)}
\end{Verbatim}


\begin{Verbatim}[commandchars=\\\{\}]
{\color{outcolor}Out[{\color{outcolor}5}]:}             LIMIT\_BAL      BILL\_AMT1      BILL\_AMT2     BILL\_AMT3  \textbackslash{}
        count    30000.000000   30000.000000   30000.000000  3.000000e+04   
        mean    167484.322667   51223.330900   49179.075167  4.701315e+04   
        std     129747.661567   73635.860576   71173.768783  6.934939e+04   
        min      10000.000000 -165580.000000  -69777.000000 -1.572640e+05   
        25\%      50000.000000    3558.750000    2984.750000  2.666250e+03   
        50\%     140000.000000   22381.500000   21200.000000  2.008850e+04   
        75\%     240000.000000   67091.000000   64006.250000  6.016475e+04   
        max    1000000.000000  964511.000000  983931.000000  1.664089e+06   
        
                   BILL\_AMT4      BILL\_AMT5      BILL\_AMT6       PAY\_AMT1  \textbackslash{}
        count   30000.000000   30000.000000   30000.000000   30000.000000   
        mean    43262.948967   40311.400967   38871.760400    5663.580500   
        std     64332.856134   60797.155770   59554.107537   16563.280354   
        min   -170000.000000  -81334.000000 -339603.000000       0.000000   
        25\%      2326.750000    1763.000000    1256.000000    1000.000000   
        50\%     19052.000000   18104.500000   17071.000000    2100.000000   
        75\%     54506.000000   50190.500000   49198.250000    5006.000000   
        max    891586.000000  927171.000000  961664.000000  873552.000000   
        
                   PAY\_AMT2      PAY\_AMT3       PAY\_AMT4       PAY\_AMT5       PAY\_AMT6  
        count  3.000000e+04   30000.00000   30000.000000   30000.000000   30000.000000  
        mean   5.921163e+03    5225.68150    4826.076867    4799.387633    5215.502567  
        std    2.304087e+04   17606.96147   15666.159744   15278.305679   17777.465775  
        min    0.000000e+00       0.00000       0.000000       0.000000       0.000000  
        25\%    8.330000e+02     390.00000     296.000000     252.500000     117.750000  
        50\%    2.009000e+03    1800.00000    1500.000000    1500.000000    1500.000000  
        75\%    5.000000e+03    4505.00000    4013.250000    4031.500000    4000.000000  
        max    1.684259e+06  896040.00000  621000.000000  426529.000000  528666.000000  
\end{Verbatim}
            
    The tables above show summary statistics for all the variables in the
data set.

Some Noted Observations: - We can see that the average person is a 35
year old woman who graduated school and pays her bills on time. -
Continuous variables appear right-skewed with large ranges. - Billed
amounts seem to be increasing with time but the pattern appears less
clear with payment amounts.

    \begin{Verbatim}[commandchars=\\\{\}]
{\color{incolor}In [{\color{incolor}6}]:} \PY{o}{\PYZpc{}}\PY{k}{matplotlib} inline
        \PY{c+c1}{\PYZsh{} find the percentage of people who were default}
        \PY{n}{percentDefault} \PY{o}{=} \PY{n+nb}{float}\PY{p}{(}\PY{n+nb}{len}\PY{p}{(}\PY{n}{df}\PY{p}{[}\PY{n}{df}\PY{o}{.}\PY{n}{default} \PY{o}{!=} \PY{l+m+mi}{0}\PY{p}{]}\PY{p}{)}\PY{p}{)}\PY{o}{/}\PY{n+nb}{len}\PY{p}{(}\PY{n}{df}\PY{p}{)} \PY{o}{*} \PY{l+m+mi}{100}
        \PY{n+nb}{print} \PY{p}{(}\PY{n}{percentDefault}\PY{p}{)}
\end{Verbatim}


    \begin{Verbatim}[commandchars=\\\{\}]
22.12

    \end{Verbatim}

    Overall, the percentage of clients in the data set who defaulted on
their credit cards is 22.12\%. This matches the mean from the data frame
describe() function in the table above. It gives us an indication of the
robustness of the data set in terms of the variable we will be
predicting.

    \begin{Verbatim}[commandchars=\\\{\}]
{\color{incolor}In [{\color{incolor}1}]:} \PY{p}{(}\PY{l+m+mf}{130109.65}\PY{o}{\PYZhy{}}\PY{l+m+mf}{178099.72}\PY{p}{)}\PY{o}{/}\PY{l+m+mf}{178099.72}
\end{Verbatim}


\begin{Verbatim}[commandchars=\\\{\}]
{\color{outcolor}Out[{\color{outcolor}1}]:} -0.26945617881937156
\end{Verbatim}
            
    The average credit limit (LIMIT\_BAL) for clients who defaulted is
26.94\% lower than those who did not. The bank may be successfully
limiting credit to those it deems riskier.

History of delinquency in the above table (PAY\_0 - PAY\_6) confirms
that late payments seem to be related to default.

    \begin{Verbatim}[commandchars=\\\{\}]
{\color{incolor}In [{\color{incolor}9}]:} \PY{n}{df}\PY{o}{.}\PY{n}{groupby}\PY{p}{(}\PY{n}{by}\PY{o}{=}\PY{n}{df}\PY{o}{.}\PY{n}{EDUCATION}\PY{p}{)}\PY{o}{.}\PY{n}{mean}\PY{p}{(}\PY{p}{)}
\end{Verbatim}


\begin{Verbatim}[commandchars=\\\{\}]
{\color{outcolor}Out[{\color{outcolor}9}]:}                LIMIT\_BAL       SEX  MARRIAGE        AGE     PAY\_0     PAY\_2  \textbackslash{}
        EDUCATION                                                                     
        0          217142.857143  1.428571  1.714286  38.857143 -0.500000 -1.000000   
        1          212956.069910  1.588663  1.652338  34.231838 -0.233916 -0.408125   
        2          147062.437634  1.616964  1.523022  34.722096  0.102210  0.022523   
        3          126550.270490  1.595282  1.421192  40.299980  0.132805  0.040879   
        4          220894.308943  1.658537  1.601626  33.853659 -0.504065 -0.772358   
        5          168164.285714  1.660714  1.475000  35.600000 -0.121429 -0.303571   
        6          148235.294118  1.509804  1.490196  43.901961 -0.176471 -0.313725   
        
                      PAY\_3     PAY\_4     PAY\_5     PAY\_6    {\ldots}        BILL\_AMT4  \textbackslash{}
        EDUCATION                                            {\ldots}                    
        0         -0.928571 -0.857143 -1.071429 -1.357143    {\ldots}     13350.214286   
        1         -0.425886 -0.461974 -0.479074 -0.485971    {\ldots}     42931.065187   
        2         -0.018532 -0.083036 -0.141411 -0.170848    {\ldots}     44748.779758   
        3          0.002644 -0.066504 -0.139313 -0.183649    {\ldots}     38718.582266   
        4         -0.764228 -0.813008 -0.780488 -0.739837    {\ldots}     39570.268293   
        5         -0.375000 -0.375000 -0.389286 -0.521429    {\ldots}     62275.767857   
        6         -0.372549 -0.411765 -0.509804 -0.647059    {\ldots}     54259.490196   
        
                      BILL\_AMT5     BILL\_AMT6     PAY\_AMT1      PAY\_AMT2     PAY\_AMT3  \textbackslash{}
        EDUCATION                                                                       
        0           7409.071429   5272.928571  5945.785714  13030.928571  8825.142857   
        1          40388.891261  38668.076051  6780.933585   7306.622201  6560.585735   
        2          41588.566287  40431.943835  5080.463293   5106.711333  4556.800000   
        3          35957.469392  34704.597315  4866.397397   5053.454139  3964.056742   
        4          33840.113821  32136.130081  5450.512195   6555.008130  9990.626016   
        5          53568.014286  46083.860714  5970.714286   8912.921429  7718.510714   
        6          44510.745098  39578.509804  9780.450980   6176.431373  7644.941176   
        
                      PAY\_AMT4      PAY\_AMT5      PAY\_AMT6   default  
        EDUCATION                                                     
        0          3620.571429   2541.714286   3007.214286  0.000000  
        1          5804.565612   5776.562211   6422.554842  0.192348  
        2          4375.387313   4452.678689   4716.487028  0.237349  
        3          3992.658532   3599.658938   3825.749034  0.251576  
        4          5104.861789   5991.642276   4284.967480  0.056911  
        5          4927.332143   4633.246429   7772.114286  0.064286  
        6          5179.490196  11691.137255  14773.901961  0.156863  
        
        [7 rows x 23 columns]
\end{Verbatim}
            
    Those with a only high school education have the highest default rate at
25.15\%. The "Other/undefined" categories are the lowest.

    \subsection{Visualize Attributes}\label{visualize-attributes}

    \begin{Verbatim}[commandchars=\\\{\}]
{\color{incolor}In [{\color{incolor}10}]:} \PY{n}{df}\PY{o}{.}\PY{n}{groupby}\PY{p}{(}\PY{n}{by}\PY{o}{=}\PY{n}{df}\PY{o}{.}\PY{n}{SEX}\PY{p}{)}\PY{o}{.}\PY{n}{mean}\PY{p}{(}\PY{p}{)}
\end{Verbatim}


\begin{Verbatim}[commandchars=\\\{\}]
{\color{outcolor}Out[{\color{outcolor}10}]:}          LIMIT\_BAL  EDUCATION  MARRIAGE        AGE     PAY\_0     PAY\_2  \textbackslash{}
         SEX                                                                      
         1    163519.825034   1.839250  1.572090  36.519431  0.063257 -0.029189   
         2    170086.462014   1.862246  1.538593  34.806868 -0.069181 -0.202407   
         
                 PAY\_3     PAY\_4     PAY\_5     PAY\_6    {\ldots}        BILL\_AMT4  \textbackslash{}
         SEX                                            {\ldots}                    
         1   -0.068557 -0.133832 -0.189182 -0.228634    {\ldots}     45000.331090   
         2   -0.230289 -0.277661 -0.316751 -0.332100    {\ldots}     42122.600099   
         
                 BILL\_AMT5     BILL\_AMT6     PAY\_AMT1     PAY\_AMT2     PAY\_AMT3  \textbackslash{}
         SEX                                                                      
         1    41587.504963  40101.775320  5668.537264  5960.720138  5412.506057   
         2    39473.816807  38064.427286  5660.327076  5895.200088  5103.057255   
         
                 PAY\_AMT4     PAY\_AMT5     PAY\_AMT6   default  
         SEX                                                   
         1    4869.177995  4830.827052  5276.196753  0.241672  
         2    4797.786992  4778.752043  5175.665305  0.207763  
         
         [2 rows x 23 columns]
\end{Verbatim}
            
    \begin{Verbatim}[commandchars=\\\{\}]
{\color{incolor}In [{\color{incolor}42}]:} \PY{n}{warnings}\PY{o}{.}\PY{n}{simplefilter}\PY{p}{(}\PY{l+s+s1}{\PYZsq{}}\PY{l+s+s1}{ignore}\PY{l+s+s1}{\PYZsq{}}\PY{p}{,} \PY{n+ne}{DeprecationWarning}\PY{p}{)}
         \PY{o}{\PYZpc{}}\PY{k}{matplotlib} inline
         \PY{n}{Default\PYZus{}counts} \PY{o}{=} \PY{n}{pd}\PY{o}{.}\PY{n}{crosstab}\PY{p}{(}\PY{p}{[}\PY{n}{df}\PY{p}{[}\PY{l+s+s1}{\PYZsq{}}\PY{l+s+s1}{SEX}\PY{l+s+s1}{\PYZsq{}}\PY{p}{]}\PY{p}{]}\PY{p}{,} \PY{n}{df}\PY{o}{.}\PY{n}{default}\PY{o}{.}\PY{n}{astype}\PY{p}{(}\PY{n+nb}{bool}\PY{p}{)}\PY{p}{)}
         \PY{c+c1}{\PYZsh{} Default\PYZus{}counts.plot(kind=\PYZsq{}bar\PYZsq{}, stacked=True, color=[\PYZsq{}grey\PYZsq{},\PYZsq{}blue\PYZsq{}])}
         \PY{n}{Default\PYZus{}counts}\PY{o}{.}\PY{n}{plot}\PY{p}{(}\PY{n}{kind}\PY{o}{=}\PY{l+s+s1}{\PYZsq{}}\PY{l+s+s1}{bar}\PY{l+s+s1}{\PYZsq{}}\PY{p}{,} \PY{n}{stacked}\PY{o}{=}\PY{k+kc}{True}\PY{p}{)}
         
         \PY{c+c1}{\PYZsh{} divide the counts to get rates}
         \PY{n}{Default\PYZus{}rate} \PY{o}{=} \PY{n}{Default\PYZus{}counts}\PY{o}{.}\PY{n}{div}\PY{p}{(}\PY{n}{Default\PYZus{}counts}\PY{o}{.}\PY{n}{sum}\PY{p}{(}\PY{l+m+mi}{1}\PY{p}{)}\PY{o}{.}\PY{n}{astype}\PY{p}{(}\PY{n+nb}{float}\PY{p}{)}\PY{p}{,}\PY{n}{axis}\PY{o}{=}\PY{l+m+mi}{0}\PY{p}{)}
         \PY{c+c1}{\PYZsh{} Default\PYZus{}rate.plot(kind=\PYZsq{}barh\PYZsq{}, stacked=True, color=[\PYZsq{}grey\PYZsq{},\PYZsq{}blue\PYZsq{}])}
         \PY{n}{Default\PYZus{}rate}\PY{o}{.}\PY{n}{plot}\PY{p}{(}\PY{n}{kind}\PY{o}{=}\PY{l+s+s1}{\PYZsq{}}\PY{l+s+s1}{barh}\PY{l+s+s1}{\PYZsq{}}\PY{p}{,} \PY{n}{stacked}\PY{o}{=}\PY{k+kc}{True}\PY{p}{)}
\end{Verbatim}


\begin{Verbatim}[commandchars=\\\{\}]
{\color{outcolor}Out[{\color{outcolor}42}]:} <matplotlib.axes.\_subplots.AxesSubplot at 0x1d458abc160>
\end{Verbatim}
            
    \begin{center}
    \adjustimage{max size={0.9\linewidth}{0.9\paperheight}}{panek_tanyuk_wall_davieau_lab1_files/panek_tanyuk_wall_davieau_lab1_21_1.png}
    \end{center}
    { \hspace*{\fill} \\}
    
    \begin{center}
    \adjustimage{max size={0.9\linewidth}{0.9\paperheight}}{panek_tanyuk_wall_davieau_lab1_files/panek_tanyuk_wall_davieau_lab1_21_2.png}
    \end{center}
    { \hspace*{\fill} \\}
    
    It satisfies our curiosity that 21\% of women default while 24\% of men
default. It is also interesting that 60\% of the data set is women. It
is not clear from these statistics alone whether gender is a meaningful
factor in estimating default rates after other variables are considered.

    \begin{Verbatim}[commandchars=\\\{\}]
{\color{incolor}In [{\color{incolor}12}]:} \PY{n}{a} \PY{o}{=} \PY{n}{df}\PY{o}{.}\PY{n}{boxplot}\PY{p}{(}\PY{n}{column}\PY{o}{=}\PY{l+s+s1}{\PYZsq{}}\PY{l+s+s1}{LIMIT\PYZus{}BAL}\PY{l+s+s1}{\PYZsq{}}\PY{p}{,} \PY{n}{by} \PY{o}{=} \PY{l+s+s1}{\PYZsq{}}\PY{l+s+s1}{default}\PY{l+s+s1}{\PYZsq{}}\PY{p}{)}
         \PY{n}{a}\PY{o}{.}\PY{n}{set\PYZus{}yscale}
\end{Verbatim}


\begin{Verbatim}[commandchars=\\\{\}]
{\color{outcolor}Out[{\color{outcolor}12}]:} <bound method \_AxesBase.set\_yscale of <matplotlib.axes.\_subplots.AxesSubplot object at 0x000001D43EA18240>>
\end{Verbatim}
            
    \begin{center}
    \adjustimage{max size={0.9\linewidth}{0.9\paperheight}}{panek_tanyuk_wall_davieau_lab1_files/panek_tanyuk_wall_davieau_lab1_23_1.png}
    \end{center}
    { \hspace*{\fill} \\}
    
    The boxplot above shows that credit limits (LIMIT\_BAL) tend to be lower
overall for those who default than those who don't. It is not just a
matter of the average being 26.94\% lower as noted earlier.

This could be interesting because the bank may be successfully managing
its risk through limits, which could indicate factors at work in the
data other than the attributes provided. The limits could be set on
variables not provided, and changes in their credit granting processes
could impact the default rates without change in the attributes
provided. It also shows that the credit limit could provide value as an
explanatory variable.

    \begin{Verbatim}[commandchars=\\\{\}]
{\color{incolor}In [{\color{incolor}13}]:} \PY{n}{b} \PY{o}{=} \PY{n}{df}\PY{o}{.}\PY{n}{boxplot}\PY{p}{(}\PY{n}{column}\PY{o}{=}\PY{l+s+s1}{\PYZsq{}}\PY{l+s+s1}{LIMIT\PYZus{}BAL}\PY{l+s+s1}{\PYZsq{}}\PY{p}{)}
         \PY{n}{b}\PY{o}{.}\PY{n}{set\PYZus{}yscale}
\end{Verbatim}


\begin{Verbatim}[commandchars=\\\{\}]
{\color{outcolor}Out[{\color{outcolor}13}]:} <bound method \_AxesBase.set\_yscale of <matplotlib.axes.\_subplots.AxesSubplot object at 0x000001D43C7E84A8>>
\end{Verbatim}
            
    \begin{center}
    \adjustimage{max size={0.9\linewidth}{0.9\paperheight}}{panek_tanyuk_wall_davieau_lab1_files/panek_tanyuk_wall_davieau_lab1_25_1.png}
    \end{center}
    { \hspace*{\fill} \\}
    
    From the boxplot above we see that credit limits are right-skewed. The
max for the top 90\% of the data is approximately 500,000 a median of
140,000 and a potential outlier at 1 million.

We believe that the 1 million point is valid because billed amounts are
consistent with that limit. We also checked some of the others and
didn't see any potential data issues.

We may want to be mindful of the extreme spread in credit limits. Such a
spread may create modelling issues and we may want to cluster customers
into similar groups.

    \begin{Verbatim}[commandchars=\\\{\}]
{\color{incolor}In [{\color{incolor}14}]:} \PY{n}{c} \PY{o}{=} \PY{n}{df}\PY{o}{.}\PY{n}{boxplot}\PY{p}{(}\PY{n}{column}\PY{o}{=}\PY{l+s+s1}{\PYZsq{}}\PY{l+s+s1}{AGE}\PY{l+s+s1}{\PYZsq{}}\PY{p}{)}
         \PY{n}{c}\PY{o}{.}\PY{n}{set\PYZus{}yscale}
\end{Verbatim}


\begin{Verbatim}[commandchars=\\\{\}]
{\color{outcolor}Out[{\color{outcolor}14}]:} <bound method \_AxesBase.set\_yscale of <matplotlib.axes.\_subplots.AxesSubplot object at 0x000001D43C950278>>
\end{Verbatim}
            
    \begin{center}
    \adjustimage{max size={0.9\linewidth}{0.9\paperheight}}{panek_tanyuk_wall_davieau_lab1_files/panek_tanyuk_wall_davieau_lab1_27_1.png}
    \end{center}
    { \hspace*{\fill} \\}
    
    The age variable seems pretty clean. 90\% of the observations are
between 21 and 60. Mean age is 35 and the median is 34. The skew doesn't
seem to dominate most of the data.

    \begin{Verbatim}[commandchars=\\\{\}]
{\color{incolor}In [{\color{incolor}15}]:} \PY{k+kn}{from} \PY{n+nn}{pandas}\PY{n+nn}{.}\PY{n+nn}{plotting} \PY{k}{import} \PY{n}{scatter\PYZus{}matrix}
         \PY{n}{ax}\PY{o}{=}\PY{n}{scatter\PYZus{}matrix}\PY{p}{(}\PY{n}{df}\PY{p}{[}\PY{p}{[}\PY{l+s+s2}{\PYZdq{}}\PY{l+s+s2}{BILL\PYZus{}AMT1}\PY{l+s+s2}{\PYZdq{}}\PY{p}{,}\PY{l+s+s2}{\PYZdq{}}\PY{l+s+s2}{BILL\PYZus{}AMT2}\PY{l+s+s2}{\PYZdq{}}\PY{p}{,}\PY{l+s+s2}{\PYZdq{}}\PY{l+s+s2}{BILL\PYZus{}AMT3}\PY{l+s+s2}{\PYZdq{}}\PY{p}{,}\PY{l+s+s2}{\PYZdq{}}\PY{l+s+s2}{BILL\PYZus{}AMT4}\PY{l+s+s2}{\PYZdq{}}\PY{p}{,}\PY{l+s+s2}{\PYZdq{}}\PY{l+s+s2}{BILL\PYZus{}AMT5}\PY{l+s+s2}{\PYZdq{}}\PY{p}{,}\PY{l+s+s2}{\PYZdq{}}\PY{l+s+s2}{BILL\PYZus{}AMT6}\PY{l+s+s2}{\PYZdq{}}\PY{p}{]}\PY{p}{]}\PY{p}{,}\PY{n}{figsize}\PY{o}{=}\PY{p}{(}\PY{l+m+mi}{15}\PY{p}{,}\PY{l+m+mi}{10}\PY{p}{)}\PY{p}{)}
\end{Verbatim}


    \begin{center}
    \adjustimage{max size={0.9\linewidth}{0.9\paperheight}}{panek_tanyuk_wall_davieau_lab1_files/panek_tanyuk_wall_davieau_lab1_29_0.png}
    \end{center}
    { \hspace*{\fill} \\}
    
    We see on the scatter matrix high correlation between amounts of bill
statements. All BILL\_AMT distributions are right skewed. There are also
indications of outliers. We looked at a small sample, and saw no reason
to question the data.

We may use a log transform for this data in subsequent analysis.

    \begin{Verbatim}[commandchars=\\\{\}]
{\color{incolor}In [{\color{incolor}16}]:} \PY{n}{vars\PYZus{}to\PYZus{}plot\PYZus{}separate} \PY{o}{=} \PY{p}{[}\PY{p}{[}\PY{l+s+s1}{\PYZsq{}}\PY{l+s+s1}{PAY\PYZus{}AMT1}\PY{l+s+s1}{\PYZsq{}}\PY{p}{]}\PY{p}{,}\PY{p}{[}\PY{l+s+s1}{\PYZsq{}}\PY{l+s+s1}{PAY\PYZus{}AMT2}\PY{l+s+s1}{\PYZsq{}}\PY{p}{]}\PY{p}{,}\PY{p}{[}\PY{l+s+s1}{\PYZsq{}}\PY{l+s+s1}{PAY\PYZus{}AMT3}\PY{l+s+s1}{\PYZsq{}}\PY{p}{]}\PY{p}{,}\PY{p}{[}\PY{l+s+s1}{\PYZsq{}}\PY{l+s+s1}{PAY\PYZus{}AMT4}\PY{l+s+s1}{\PYZsq{}}\PY{p}{]}\PY{p}{,}\PY{p}{[}\PY{l+s+s1}{\PYZsq{}}\PY{l+s+s1}{PAY\PYZus{}AMT5}\PY{l+s+s1}{\PYZsq{}}\PY{p}{]}\PY{p}{,}\PY{p}{[}\PY{l+s+s1}{\PYZsq{}}\PY{l+s+s1}{PAY\PYZus{}AMT6}\PY{l+s+s1}{\PYZsq{}}\PY{p}{]}\PY{p}{]}
         \PY{n}{plt}\PY{o}{.}\PY{n}{figure}\PY{p}{(}\PY{n}{figsize}\PY{o}{=}\PY{p}{(}\PY{l+m+mi}{10}\PY{p}{,}\PY{l+m+mi}{30}\PY{p}{)}\PY{p}{)}
         \PY{k}{for} \PY{n}{index}\PY{p}{,} \PY{n}{plot\PYZus{}vars} \PY{o+ow}{in} \PY{n+nb}{enumerate} \PY{p}{(}\PY{n}{vars\PYZus{}to\PYZus{}plot\PYZus{}separate}\PY{p}{)}\PY{p}{:}
             \PY{n}{plt}\PY{o}{.}\PY{n}{subplot}\PY{p}{(}\PY{n+nb}{len}\PY{p}{(}\PY{n}{vars\PYZus{}to\PYZus{}plot\PYZus{}separate}\PY{p}{)}\PY{o}{/}\PY{l+m+mi}{2}\PY{p}{,}
                        \PY{l+m+mi}{2}\PY{p}{,}
                        \PY{n}{index}\PY{o}{+}\PY{l+m+mi}{1}\PY{p}{)}
             \PY{n}{ax}\PY{o}{=}\PY{n}{df}\PY{o}{.}\PY{n}{boxplot}\PY{p}{(}\PY{n}{column}\PY{o}{=}\PY{n}{plot\PYZus{}vars}\PY{p}{)}
             
         \PY{n}{plt}\PY{o}{.}\PY{n}{show}\PY{p}{(}\PY{p}{)}
\end{Verbatim}


    \begin{center}
    \adjustimage{max size={0.9\linewidth}{0.9\paperheight}}{panek_tanyuk_wall_davieau_lab1_files/panek_tanyuk_wall_davieau_lab1_31_0.png}
    \end{center}
    { \hspace*{\fill} \\}
    
    As observed previously the range of values in the amount of payments
made is extremely wide. This makes visualization and potentially
modeling difficult. Again log or some other transforms may be necessary
but we will keep the scale for the purposes of visualization.

    \begin{Verbatim}[commandchars=\\\{\}]
{\color{incolor}In [{\color{incolor}17}]:} \PY{n}{fig}\PY{p}{,} \PY{n}{axs} \PY{o}{=} \PY{n}{plt}\PY{o}{.}\PY{n}{subplots}\PY{p}{(}\PY{l+m+mi}{2}\PY{p}{,} \PY{l+m+mi}{3}\PY{p}{)}
         \PY{n}{axs}\PY{p}{[}\PY{l+m+mi}{0}\PY{p}{,} \PY{l+m+mi}{0}\PY{p}{]}\PY{o}{.}\PY{n}{boxplot}\PY{p}{(}\PY{n}{df}\PY{o}{.}\PY{n}{PAY\PYZus{}AMT1}\PY{p}{,} \PY{l+m+mi}{0}\PY{p}{,} \PY{l+s+s1}{\PYZsq{}}\PY{l+s+s1}{\PYZsq{}}\PY{p}{)}
         \PY{n}{axs}\PY{p}{[}\PY{l+m+mi}{0}\PY{p}{,} \PY{l+m+mi}{0}\PY{p}{]}\PY{o}{.}\PY{n}{set\PYZus{}title}\PY{p}{(}\PY{l+s+s2}{\PYZdq{}}\PY{l+s+s2}{PAY\PYZus{}AMT1 no outlier}\PY{l+s+s2}{\PYZdq{}}\PY{p}{)}
         \PY{n}{axs}\PY{p}{[}\PY{l+m+mi}{0}\PY{p}{,} \PY{l+m+mi}{1}\PY{p}{]}\PY{o}{.}\PY{n}{boxplot}\PY{p}{(}\PY{n}{df}\PY{o}{.}\PY{n}{PAY\PYZus{}AMT2}\PY{p}{,} \PY{l+m+mi}{0}\PY{p}{,} \PY{l+s+s1}{\PYZsq{}}\PY{l+s+s1}{\PYZsq{}}\PY{p}{)}
         \PY{n}{axs}\PY{p}{[}\PY{l+m+mi}{0}\PY{p}{,} \PY{l+m+mi}{1}\PY{p}{]}\PY{o}{.}\PY{n}{set\PYZus{}title}\PY{p}{(}\PY{l+s+s2}{\PYZdq{}}\PY{l+s+s2}{PAY\PYZus{}AMT2 no outlier}\PY{l+s+s2}{\PYZdq{}}\PY{p}{)}
         \PY{n}{axs}\PY{p}{[}\PY{l+m+mi}{0}\PY{p}{,} \PY{l+m+mi}{2}\PY{p}{]}\PY{o}{.}\PY{n}{boxplot}\PY{p}{(}\PY{n}{df}\PY{o}{.}\PY{n}{PAY\PYZus{}AMT3}\PY{p}{,} \PY{l+m+mi}{0}\PY{p}{,} \PY{l+s+s1}{\PYZsq{}}\PY{l+s+s1}{\PYZsq{}}\PY{p}{)}
         \PY{n}{axs}\PY{p}{[}\PY{l+m+mi}{0}\PY{p}{,} \PY{l+m+mi}{2}\PY{p}{]}\PY{o}{.}\PY{n}{set\PYZus{}title}\PY{p}{(}\PY{l+s+s2}{\PYZdq{}}\PY{l+s+s2}{PAY\PYZus{}AMT3 no outlier}\PY{l+s+s2}{\PYZdq{}}\PY{p}{)}
         \PY{n}{axs}\PY{p}{[}\PY{l+m+mi}{1}\PY{p}{,} \PY{l+m+mi}{0}\PY{p}{]}\PY{o}{.}\PY{n}{boxplot}\PY{p}{(}\PY{n}{df}\PY{o}{.}\PY{n}{PAY\PYZus{}AMT4}\PY{p}{,} \PY{l+m+mi}{0}\PY{p}{,} \PY{l+s+s1}{\PYZsq{}}\PY{l+s+s1}{\PYZsq{}}\PY{p}{)}
         \PY{n}{axs}\PY{p}{[}\PY{l+m+mi}{1}\PY{p}{,} \PY{l+m+mi}{0}\PY{p}{]}\PY{o}{.}\PY{n}{set\PYZus{}title}\PY{p}{(}\PY{l+s+s2}{\PYZdq{}}\PY{l+s+s2}{PAY\PYZus{}AMT4 no outlier}\PY{l+s+s2}{\PYZdq{}}\PY{p}{)}
         \PY{n}{axs}\PY{p}{[}\PY{l+m+mi}{1}\PY{p}{,} \PY{l+m+mi}{1}\PY{p}{]}\PY{o}{.}\PY{n}{boxplot}\PY{p}{(}\PY{n}{df}\PY{o}{.}\PY{n}{PAY\PYZus{}AMT4}\PY{p}{,} \PY{l+m+mi}{0}\PY{p}{,} \PY{l+s+s1}{\PYZsq{}}\PY{l+s+s1}{\PYZsq{}}\PY{p}{)}
         \PY{n}{axs}\PY{p}{[}\PY{l+m+mi}{1}\PY{p}{,} \PY{l+m+mi}{1}\PY{p}{]}\PY{o}{.}\PY{n}{set\PYZus{}title}\PY{p}{(}\PY{l+s+s2}{\PYZdq{}}\PY{l+s+s2}{PAY\PYZus{}AMT5 no outlier}\PY{l+s+s2}{\PYZdq{}}\PY{p}{)}
         \PY{n}{axs}\PY{p}{[}\PY{l+m+mi}{1}\PY{p}{,} \PY{l+m+mi}{2}\PY{p}{]}\PY{o}{.}\PY{n}{boxplot}\PY{p}{(}\PY{n}{df}\PY{o}{.}\PY{n}{PAY\PYZus{}AMT6}\PY{p}{,} \PY{l+m+mi}{0}\PY{p}{,} \PY{l+s+s1}{\PYZsq{}}\PY{l+s+s1}{\PYZsq{}}\PY{p}{)}
         \PY{n}{axs}\PY{p}{[}\PY{l+m+mi}{1}\PY{p}{,} \PY{l+m+mi}{2}\PY{p}{]}\PY{o}{.}\PY{n}{set\PYZus{}title}\PY{p}{(}\PY{l+s+s2}{\PYZdq{}}\PY{l+s+s2}{PAY\PYZus{}AMT6 no outlier}\PY{l+s+s2}{\PYZdq{}}\PY{p}{)}
\end{Verbatim}


\begin{Verbatim}[commandchars=\\\{\}]
{\color{outcolor}Out[{\color{outcolor}17}]:} Text(0.5,1,'PAY\_AMT6 no outlier')
\end{Verbatim}
            
    \begin{center}
    \adjustimage{max size={0.9\linewidth}{0.9\paperheight}}{panek_tanyuk_wall_davieau_lab1_files/panek_tanyuk_wall_davieau_lab1_33_1.png}
    \end{center}
    { \hspace*{\fill} \\}
    
    Removal of outliers above \$10,000 (about the 90th percentile) makes the
box-plots more interpretable. The plots look similar from month to
month, but the skew seems to be decreasing with time. This point will be
important for modeling.

    \subsection{Explore Joint Attributes}\label{explore-joint-attributes}

\paragraph{\texorpdfstring{\emph{\emph{Overview}}}{Overview}}\label{overview}

Amount Billed, Payment Amount, and Pay(Delinquency Status) are time
series. We examined for autocorrelation by reviewing scatterplots,
correlation matrices and cross-tabs.

There are assumed structural relationships between some of the
attributes: 1. The amount of a payment is likely related to the Amount
Billed in the prior month 2. The Amount Billed is likely limited by the
Credit Limit

    \paragraph{Serial Correlation Analysis:
Payments}\label{serial-correlation-analysis-payments}

    \begin{Verbatim}[commandchars=\\\{\}]
{\color{incolor}In [{\color{incolor}18}]:} \PY{o}{\PYZpc{}}\PY{k}{matplotlib} inline
         \PY{n}{sns}\PY{o}{.}\PY{n}{set}\PY{p}{(}\PY{p}{)}
         \PY{n}{g} \PY{o}{=} \PY{n}{sns}\PY{o}{.}\PY{n}{pairplot}\PY{p}{(}\PY{n}{df}\PY{p}{[}\PY{n}{Payments}\PY{p}{]}\PY{p}{,}\PY{n}{hue} \PY{o}{=} \PY{l+s+s1}{\PYZsq{}}\PY{l+s+s1}{default}\PY{l+s+s1}{\PYZsq{}}\PY{p}{,}\PY{n}{size} \PY{o}{=} \PY{l+m+mi}{2}\PY{p}{)}
\end{Verbatim}


    \begin{center}
    \adjustimage{max size={0.9\linewidth}{0.9\paperheight}}{panek_tanyuk_wall_davieau_lab1_files/panek_tanyuk_wall_davieau_lab1_37_0.png}
    \end{center}
    { \hspace*{\fill} \\}
    
    \begin{Verbatim}[commandchars=\\\{\}]
{\color{incolor}In [{\color{incolor}19}]:} \PY{n}{df}\PY{p}{[}\PY{n}{Payments}\PY{p}{]}\PY{o}{.}\PY{n}{corr}\PY{p}{(}\PY{p}{)}
\end{Verbatim}


\begin{Verbatim}[commandchars=\\\{\}]
{\color{outcolor}Out[{\color{outcolor}19}]:}           PAY\_AMT1  PAY\_AMT2  PAY\_AMT3  PAY\_AMT4  PAY\_AMT5  PAY\_AMT6   default
         PAY\_AMT1  1.000000  0.285576  0.252191  0.199558  0.148459  0.185735 -0.072929
         PAY\_AMT2  0.285576  1.000000  0.244770  0.180107  0.180908  0.157634 -0.058579
         PAY\_AMT3  0.252191  0.244770  1.000000  0.216325  0.159214  0.162740 -0.056250
         PAY\_AMT4  0.199558  0.180107  0.216325  1.000000  0.151830  0.157834 -0.056827
         PAY\_AMT5  0.148459  0.180908  0.159214  0.151830  1.000000  0.154896 -0.055124
         PAY\_AMT6  0.185735  0.157634  0.162740  0.157834  0.154896  1.000000 -0.053183
         default  -0.072929 -0.058579 -0.056250 -0.056827 -0.055124 -0.053183  1.000000
\end{Verbatim}
            
    A visual review of the scatter plots above does not show obvious serial
correlation in payments. However, correlation coefficients approaching
30\% in certain cases as indicated in the table above indicate that this
is something we need to consider in the modelling phase.

    \paragraph{Serial Correlation Analysis: Billed Amounts and Credit
Limits}\label{serial-correlation-analysis-billed-amounts-and-credit-limits}

    \begin{Verbatim}[commandchars=\\\{\}]
{\color{incolor}In [{\color{incolor}20}]:} \PY{o}{\PYZpc{}}\PY{k}{matplotlib} inline
         \PY{n}{sns}\PY{o}{.}\PY{n}{set}\PY{p}{(}\PY{p}{)}
         \PY{n}{sns}\PY{o}{.}\PY{n}{pairplot}\PY{p}{(}\PY{n}{df}\PY{p}{[}\PY{n}{Bills}\PY{p}{]}\PY{p}{,}\PY{n}{hue} \PY{o}{=} \PY{l+s+s1}{\PYZsq{}}\PY{l+s+s1}{default}\PY{l+s+s1}{\PYZsq{}}\PY{p}{,}\PY{n}{size} \PY{o}{=} \PY{l+m+mi}{2}\PY{p}{)}
\end{Verbatim}


\begin{Verbatim}[commandchars=\\\{\}]
{\color{outcolor}Out[{\color{outcolor}20}]:} <seaborn.axisgrid.PairGrid at 0x1d43ca312b0>
\end{Verbatim}
            
    \begin{center}
    \adjustimage{max size={0.9\linewidth}{0.9\paperheight}}{panek_tanyuk_wall_davieau_lab1_files/panek_tanyuk_wall_davieau_lab1_41_1.png}
    \end{center}
    { \hspace*{\fill} \\}
    
    \begin{Verbatim}[commandchars=\\\{\}]
{\color{incolor}In [{\color{incolor}21}]:} \PY{n}{df}\PY{p}{[}\PY{n}{Bills}\PY{p}{]}\PY{o}{.}\PY{n}{corr}\PY{p}{(}\PY{p}{)}
\end{Verbatim}


\begin{Verbatim}[commandchars=\\\{\}]
{\color{outcolor}Out[{\color{outcolor}21}]:}            LIMIT\_BAL  BILL\_AMT1  BILL\_AMT2  BILL\_AMT3  BILL\_AMT4  BILL\_AMT5  \textbackslash{}
         LIMIT\_BAL   1.000000   0.285430   0.278314   0.283236   0.293988   0.295562   
         BILL\_AMT1   0.285430   1.000000   0.951484   0.892279   0.860272   0.829779   
         BILL\_AMT2   0.278314   0.951484   1.000000   0.928326   0.892482   0.859778   
         BILL\_AMT3   0.283236   0.892279   0.928326   1.000000   0.923969   0.883910   
         BILL\_AMT4   0.293988   0.860272   0.892482   0.923969   1.000000   0.940134   
         BILL\_AMT5   0.295562   0.829779   0.859778   0.883910   0.940134   1.000000   
         BILL\_AMT6   0.290389   0.802650   0.831594   0.853320   0.900941   0.946197   
         default    -0.153520  -0.019644  -0.014193  -0.014076  -0.010156  -0.006760   
         
                    BILL\_AMT6   default  
         LIMIT\_BAL   0.290389 -0.153520  
         BILL\_AMT1   0.802650 -0.019644  
         BILL\_AMT2   0.831594 -0.014193  
         BILL\_AMT3   0.853320 -0.014076  
         BILL\_AMT4   0.900941 -0.010156  
         BILL\_AMT5   0.946197 -0.006760  
         BILL\_AMT6   1.000000 -0.005372  
         default    -0.005372  1.000000  
\end{Verbatim}
            
    A visual review of the scatter plots above shows clear indication of
correlation in Billed Amounts across different months. It also shows
that the Billed Amounts tend to be bound by the Credit Limit.

This is confirmed by the high correlation coefficients in the table
above. We will need to consider mitigation methods in the modelling
phase.

    \paragraph{Serial Correlation Analysis: PAY ("Payment Delinquency
Status")}\label{serial-correlation-analysis-pay-payment-delinquency-status}

    \begin{Verbatim}[commandchars=\\\{\}]
{\color{incolor}In [{\color{incolor}22}]:} \PY{c+c1}{\PYZsh{} Creating a cross\PYZhy{}tab of counts by the value of the ordinal variable to check autocerrelation in}
         \PY{c+c1}{\PYZsh{}Payment Status.}
         \PY{c+c1}{\PYZsh{} We are showing the migration of statuses from one period to the next.}
         \PY{c+c1}{\PYZsh{} The rows are the first period status and the columns are the next period status. }
         \PY{c+c1}{\PYZsh{} This requires re\PYZhy{}shaping (stacking) data.}
         
         \PY{c+c1}{\PYZsh{} First, create a single column of all Initial time period statuses,}
         \PY{c+c1}{\PYZsh{}regardless of attribute name (month).}
         \PY{n}{InitialList} \PY{o}{=} \PY{p}{[}\PY{l+s+s1}{\PYZsq{}}\PY{l+s+s1}{PAY\PYZus{}6}\PY{l+s+s1}{\PYZsq{}}\PY{p}{,}\PY{l+s+s1}{\PYZsq{}}\PY{l+s+s1}{PAY\PYZus{}5}\PY{l+s+s1}{\PYZsq{}}\PY{p}{,}\PY{l+s+s1}{\PYZsq{}}\PY{l+s+s1}{PAY\PYZus{}4}\PY{l+s+s1}{\PYZsq{}}\PY{p}{,}\PY{l+s+s1}{\PYZsq{}}\PY{l+s+s1}{PAY\PYZus{}3}\PY{l+s+s1}{\PYZsq{}}\PY{p}{,}\PY{l+s+s1}{\PYZsq{}}\PY{l+s+s1}{PAY\PYZus{}2}\PY{l+s+s1}{\PYZsq{}}\PY{p}{]}
         \PY{n}{dfFirstStatus} \PY{o}{=} \PY{n}{df}\PY{p}{[}\PY{n}{InitialList}\PY{p}{]}
         \PY{n}{StatusStacked} \PY{o}{=} \PY{n}{pd}\PY{o}{.}\PY{n}{DataFrame}\PY{p}{(}\PY{p}{\PYZob{}}\PY{l+s+s1}{\PYZsq{}}\PY{l+s+s1}{First}\PY{l+s+s1}{\PYZsq{}}\PY{p}{:}\PY{n}{dfFirstStatus}\PY{o}{.}\PY{n}{stack}\PY{p}{(}\PY{p}{)}\PY{p}{\PYZcb{}}\PY{p}{)}
         \PY{n}{StatusStacked} \PY{o}{=} \PY{n}{StatusStacked}\PY{o}{.}\PY{n}{reset\PYZus{}index}\PY{p}{(}\PY{n}{drop}\PY{o}{=}\PY{k+kc}{True}\PY{p}{)}
         
         \PY{c+c1}{\PYZsh{} Then, create a single column of all subsequent time period statuses, regardless}
         \PY{c+c1}{\PYZsh{}of attribute name (month).}
         \PY{n}{NextList} \PY{o}{=} \PY{p}{[}\PY{l+s+s1}{\PYZsq{}}\PY{l+s+s1}{PAY\PYZus{}5}\PY{l+s+s1}{\PYZsq{}}\PY{p}{,}\PY{l+s+s1}{\PYZsq{}}\PY{l+s+s1}{PAY\PYZus{}4}\PY{l+s+s1}{\PYZsq{}}\PY{p}{,}\PY{l+s+s1}{\PYZsq{}}\PY{l+s+s1}{PAY\PYZus{}3}\PY{l+s+s1}{\PYZsq{}}\PY{p}{,}\PY{l+s+s1}{\PYZsq{}}\PY{l+s+s1}{PAY\PYZus{}2}\PY{l+s+s1}{\PYZsq{}}\PY{p}{,}\PY{l+s+s1}{\PYZsq{}}\PY{l+s+s1}{PAY\PYZus{}0}\PY{l+s+s1}{\PYZsq{}}\PY{p}{]}
         \PY{n}{dfNextStatus} \PY{o}{=} \PY{n}{df}\PY{p}{[}\PY{n}{NextList}\PY{p}{]}
         \PY{n}{NextStatusStacked} \PY{o}{=} \PY{n}{pd}\PY{o}{.}\PY{n}{DataFrame}\PY{p}{(}\PY{p}{\PYZob{}}\PY{l+s+s1}{\PYZsq{}}\PY{l+s+s1}{Next}\PY{l+s+s1}{\PYZsq{}}\PY{p}{:}\PY{n}{dfNextStatus}\PY{o}{.}\PY{n}{stack}\PY{p}{(}\PY{p}{)}\PY{p}{\PYZcb{}}\PY{p}{)}
         \PY{n}{NextStatusStacked} \PY{o}{=} \PY{n}{NextStatusStacked}\PY{o}{.}\PY{n}{reset\PYZus{}index}\PY{p}{(}\PY{n}{drop}\PY{o}{=}\PY{k+kc}{True}\PY{p}{)}
         
         \PY{c+c1}{\PYZsh{} Combining the two columns into a single df.}
         \PY{n}{StatusStacked}\PY{p}{[}\PY{l+s+s1}{\PYZsq{}}\PY{l+s+s1}{Next}\PY{l+s+s1}{\PYZsq{}}\PY{p}{]} \PY{o}{=} \PY{n}{NextStatusStacked}\PY{p}{[}\PY{l+s+s1}{\PYZsq{}}\PY{l+s+s1}{Next}\PY{l+s+s1}{\PYZsq{}}\PY{p}{]}
         \PY{n}{pd}\PY{o}{.}\PY{n}{crosstab}\PY{p}{(}\PY{n}{StatusStacked}\PY{p}{[}\PY{l+s+s1}{\PYZsq{}}\PY{l+s+s1}{First}\PY{l+s+s1}{\PYZsq{}}\PY{p}{]}\PY{p}{,}\PY{n}{StatusStacked}\PY{p}{[}\PY{l+s+s1}{\PYZsq{}}\PY{l+s+s1}{Next}\PY{l+s+s1}{\PYZsq{}}\PY{p}{]}\PY{p}{)}
\end{Verbatim}


\begin{Verbatim}[commandchars=\\\{\}]
{\color{outcolor}Out[{\color{outcolor}22}]:} Next      -2     -1      0     1     2     3    4    5   6    7   8
         First                                                              
         -2     17602   1477   1091  1233   253     0    0    0   0    0   0
         -1      1904  21915   3476   621  1038     0    0    0   0    0   0
          0         4   4106  72148     6  4918     0    0    0   0    0   0
          1         0      0      0    34     0     0    0    0   0    0   0
          2        10   1306   2814  1676  9460  1031    0    0   0    0   0
          3         0     84     92   109   362   176  285    0   0    0   0
          4         0      8      8    32    85    29  106  109   0    0   0
          5         0      3      3     7    18     7   11   12  50    0   0
          6         0      1      1     2     5     1    1    3   4   45   0
          7         0      0      0     1    56     2    0    0   1  126  23
          8         0      0      0     1     3     0    1    0   0    1   3
\end{Verbatim}
            
    The table above shows the propensity of an account to remain in the same
status in some cases. We will illustrate how the table above works by
describing the first row. The table shows that there were
17,602+1,477+1,091+1,233+253=21,656 instances where an account had a Pay
Status of -2 in a month where the status in the next month is available.
Of those instances, 17,602 also had a Status of -2 in the folowing
month, and 1,477 had a Status of -1 in the following month and so on.

The table shows that accounts with initial Statuses of -2 through 2 tend
to stay in that status the following month. Status 1 seems to be a
special case discussed in \textbf{\emph{Data Quality}}. We will need to
account for this in the modelling phase.

    \begin{Verbatim}[commandchars=\\\{\}]
{\color{incolor}In [{\color{incolor}23}]:} \PY{n}{df}\PY{p}{[}\PY{n}{PayStausOnly}\PY{p}{]}\PY{o}{.}\PY{n}{apply}\PY{p}{(}\PY{n}{pd}\PY{o}{.}\PY{n}{value\PYZus{}counts}\PY{p}{)}
\end{Verbatim}


\begin{Verbatim}[commandchars=\\\{\}]
{\color{outcolor}Out[{\color{outcolor}23}]:}     PAY\_0  PAY\_2  PAY\_3  PAY\_4    PAY\_5    PAY\_6
         -2   2759   3782   4085   4348   4546.0   4895.0
         -1   5686   6050   5938   5687   5539.0   5740.0
          0  14737  15730  15764  16455  16947.0  16286.0
          1   3688     28      4      2      NaN      NaN
          2   2667   3927   3819   3159   2626.0   2766.0
          3    322    326    240    180    178.0    184.0
          4     76     99     76     69     84.0     49.0
          5     26     25     21     35     17.0     13.0
          6     11     12     23      5      4.0     19.0
          7      9     20     27     58     58.0     46.0
          8     19      1      3      2      1.0      2.0
\end{Verbatim}
            
    The table above shows that a value of 1 for this variable only appears
in meaningful amounts in the PAY\_0, the most recent month. It could be
that this indicates the creation of a new "Status" or a data issue.

    \paragraph{Serial Correlation Analysis: Billed Amounts and Payment
Amounts}\label{serial-correlation-analysis-billed-amounts-and-payment-amounts}

    \begin{Verbatim}[commandchars=\\\{\}]
{\color{incolor}In [{\color{incolor}24}]:} \PY{o}{\PYZpc{}}\PY{k}{matplotlib} inline
         \PY{n}{sns}\PY{o}{.}\PY{n}{set}\PY{p}{(}\PY{p}{)}
         \PY{n}{sns}\PY{o}{.}\PY{n}{pairplot}\PY{p}{(}\PY{n}{df}\PY{p}{[}\PY{n}{BillsAndPayments}\PY{p}{]}\PY{p}{,}\PY{n}{hue} \PY{o}{=} \PY{l+s+s1}{\PYZsq{}}\PY{l+s+s1}{default}\PY{l+s+s1}{\PYZsq{}}\PY{p}{,}\PY{n}{size} \PY{o}{=} \PY{l+m+mi}{2}\PY{p}{)}
\end{Verbatim}


\begin{Verbatim}[commandchars=\\\{\}]
{\color{outcolor}Out[{\color{outcolor}24}]:} <seaborn.axisgrid.PairGrid at 0x1d4449ee080>
\end{Verbatim}
            
    \begin{center}
    \adjustimage{max size={0.9\linewidth}{0.9\paperheight}}{panek_tanyuk_wall_davieau_lab1_files/panek_tanyuk_wall_davieau_lab1_50_1.png}
    \end{center}
    { \hspace*{\fill} \\}
    
    \begin{Verbatim}[commandchars=\\\{\}]
{\color{incolor}In [{\color{incolor}25}]:} \PY{n}{df}\PY{p}{[}\PY{n}{BillsAndPayments}\PY{p}{]}\PY{o}{.}\PY{n}{corr}\PY{p}{(}\PY{p}{)}
\end{Verbatim}


\begin{Verbatim}[commandchars=\\\{\}]
{\color{outcolor}Out[{\color{outcolor}25}]:}            BILL\_AMT1  BILL\_AMT2  BILL\_AMT3  BILL\_AMT4  BILL\_AMT5  BILL\_AMT6  \textbackslash{}
         BILL\_AMT1   1.000000   0.951484   0.892279   0.860272   0.829779   0.802650   
         BILL\_AMT2   0.951484   1.000000   0.928326   0.892482   0.859778   0.831594   
         BILL\_AMT3   0.892279   0.928326   1.000000   0.923969   0.883910   0.853320   
         BILL\_AMT4   0.860272   0.892482   0.923969   1.000000   0.940134   0.900941   
         BILL\_AMT5   0.829779   0.859778   0.883910   0.940134   1.000000   0.946197   
         BILL\_AMT6   0.802650   0.831594   0.853320   0.900941   0.946197   1.000000   
         PAY\_AMT1    0.140277   0.280365   0.244335   0.233012   0.217031   0.199965   
         PAY\_AMT2    0.099355   0.100851   0.316936   0.207564   0.181246   0.172663   
         PAY\_AMT3    0.156887   0.150718   0.130011   0.300023   0.252305   0.233770   
         PAY\_AMT4    0.158303   0.147398   0.143405   0.130191   0.293118   0.250237   
         PAY\_AMT5    0.167026   0.157957   0.179712   0.160433   0.141574   0.307729   
         PAY\_AMT6    0.179341   0.174256   0.182326   0.177637   0.164184   0.115494   
         default    -0.019644  -0.014193  -0.014076  -0.010156  -0.006760  -0.005372   
         
                    PAY\_AMT1  PAY\_AMT2  PAY\_AMT3  PAY\_AMT4  PAY\_AMT5  PAY\_AMT6  \textbackslash{}
         BILL\_AMT1  0.140277  0.099355  0.156887  0.158303  0.167026  0.179341   
         BILL\_AMT2  0.280365  0.100851  0.150718  0.147398  0.157957  0.174256   
         BILL\_AMT3  0.244335  0.316936  0.130011  0.143405  0.179712  0.182326   
         BILL\_AMT4  0.233012  0.207564  0.300023  0.130191  0.160433  0.177637   
         BILL\_AMT5  0.217031  0.181246  0.252305  0.293118  0.141574  0.164184   
         BILL\_AMT6  0.199965  0.172663  0.233770  0.250237  0.307729  0.115494   
         PAY\_AMT1   1.000000  0.285576  0.252191  0.199558  0.148459  0.185735   
         PAY\_AMT2   0.285576  1.000000  0.244770  0.180107  0.180908  0.157634   
         PAY\_AMT3   0.252191  0.244770  1.000000  0.216325  0.159214  0.162740   
         PAY\_AMT4   0.199558  0.180107  0.216325  1.000000  0.151830  0.157834   
         PAY\_AMT5   0.148459  0.180908  0.159214  0.151830  1.000000  0.154896   
         PAY\_AMT6   0.185735  0.157634  0.162740  0.157834  0.154896  1.000000   
         default   -0.072929 -0.058579 -0.056250 -0.056827 -0.055124 -0.053183   
         
                     default  
         BILL\_AMT1 -0.019644  
         BILL\_AMT2 -0.014193  
         BILL\_AMT3 -0.014076  
         BILL\_AMT4 -0.010156  
         BILL\_AMT5 -0.006760  
         BILL\_AMT6 -0.005372  
         PAY\_AMT1  -0.072929  
         PAY\_AMT2  -0.058579  
         PAY\_AMT3  -0.056250  
         PAY\_AMT4  -0.056827  
         PAY\_AMT5  -0.055124  
         PAY\_AMT6  -0.053183  
         default    1.000000  
\end{Verbatim}
            
    A visual review of the scatterplots above show that there is a potential
relationship between the Billed Amount in one month (e.g. BILL\_AMT6)
and the resulting payment (e.g. PAY\_AMT5). This is confirmed by the
correlation matrix, which indicates that the correlation between the
billed amount and the resulting payment is in the 30\% range.

    \begin{Verbatim}[commandchars=\\\{\}]
{\color{incolor}In [{\color{incolor}26}]:} \PY{c+c1}{\PYZsh{}Adding Prior Period Balances to table used for evaluating Status accuracy}
         \PY{n}{BillList} \PY{o}{=} \PY{p}{[}\PY{l+s+s1}{\PYZsq{}}\PY{l+s+s1}{BILL\PYZus{}AMT6}\PY{l+s+s1}{\PYZsq{}}\PY{p}{,}\PY{l+s+s1}{\PYZsq{}}\PY{l+s+s1}{BILL\PYZus{}AMT5}\PY{l+s+s1}{\PYZsq{}}\PY{p}{,}\PY{l+s+s1}{\PYZsq{}}\PY{l+s+s1}{BILL\PYZus{}AMT4}\PY{l+s+s1}{\PYZsq{}}\PY{p}{,}\PY{l+s+s1}{\PYZsq{}}\PY{l+s+s1}{BILL\PYZus{}AMT3}\PY{l+s+s1}{\PYZsq{}}\PY{p}{,}\PY{l+s+s1}{\PYZsq{}}\PY{l+s+s1}{BILL\PYZus{}AMT2}\PY{l+s+s1}{\PYZsq{}}\PY{p}{]}
         \PY{n}{dfPriorBill} \PY{o}{=} \PY{n}{df}\PY{p}{[}\PY{n}{BillList}\PY{p}{]}
         
         \PY{n}{BillStacked} \PY{o}{=} \PY{n}{pd}\PY{o}{.}\PY{n}{DataFrame}\PY{p}{(}\PY{p}{\PYZob{}}\PY{l+s+s1}{\PYZsq{}}\PY{l+s+s1}{Billed}\PY{l+s+s1}{\PYZsq{}}\PY{p}{:}\PY{n}{dfPriorBill}\PY{o}{.}\PY{n}{stack}\PY{p}{(}\PY{p}{)}\PY{p}{\PYZcb{}}\PY{p}{)}
         \PY{n}{BillStacked} \PY{o}{=} \PY{n}{BillStacked}\PY{o}{.}\PY{n}{reset\PYZus{}index}\PY{p}{(}\PY{n}{drop}\PY{o}{=}\PY{k+kc}{True}\PY{p}{)}
         \PY{n}{StatusStacked}\PY{p}{[}\PY{l+s+s1}{\PYZsq{}}\PY{l+s+s1}{Billed}\PY{l+s+s1}{\PYZsq{}}\PY{p}{]} \PY{o}{=} \PY{n}{BillStacked}\PY{p}{[}\PY{l+s+s1}{\PYZsq{}}\PY{l+s+s1}{Billed}\PY{l+s+s1}{\PYZsq{}}\PY{p}{]}
\end{Verbatim}


    \begin{Verbatim}[commandchars=\\\{\}]
{\color{incolor}In [{\color{incolor}27}]:} \PY{c+c1}{\PYZsh{}\PYZsh{} Checking instances of an account coded as delinquent when Billed amount was zero or less,}
         \PY{n}{ProblemCountV\PYZus{}Minus1} \PY{o}{=} \PY{l+m+mi}{0}
         \PY{n}{ProblemCountV\PYZus{}Zero} \PY{o}{=} \PY{l+m+mi}{0}
         
         \PY{c+c1}{\PYZsh{} Checking Delinquency if Status 0 is Delinquent:}
         \PY{k}{for} \PY{n}{i} \PY{o+ow}{in} \PY{n}{np}\PY{o}{.}\PY{n}{arange}\PY{p}{(}\PY{l+m+mi}{0}\PY{p}{,}\PY{n+nb}{len}\PY{p}{(}\PY{n}{StatusStacked}\PY{p}{)}\PY{p}{)}\PY{p}{:}
             \PY{k}{if} \PY{n}{StatusStacked}\PY{p}{[}\PY{l+s+s1}{\PYZsq{}}\PY{l+s+s1}{Billed}\PY{l+s+s1}{\PYZsq{}}\PY{p}{]}\PY{p}{[}\PY{n}{i}\PY{p}{]} \PY{o}{\PYZlt{}} \PY{l+m+mi}{1} \PY{o+ow}{and} \PY{n}{StatusStacked}\PY{p}{[}\PY{l+s+s1}{\PYZsq{}}\PY{l+s+s1}{Next}\PY{l+s+s1}{\PYZsq{}}\PY{p}{]}\PY{p}{[}\PY{n}{i}\PY{p}{]}\PY{o}{\PYZgt{}}\PY{o}{\PYZhy{}}\PY{l+m+mi}{1}\PY{p}{:}
                 \PY{n}{ProblemCountV\PYZus{}Minus1} \PY{o}{=} \PY{n}{ProblemCountV\PYZus{}Minus1} \PY{o}{+} \PY{l+m+mi}{1}
         
         \PY{c+c1}{\PYZsh{} Checking Delinquency if Status 1 only is Delinquent:}
         \PY{k}{for} \PY{n}{i} \PY{o+ow}{in} \PY{n}{np}\PY{o}{.}\PY{n}{arange}\PY{p}{(}\PY{l+m+mi}{0}\PY{p}{,}\PY{n+nb}{len}\PY{p}{(}\PY{n}{StatusStacked}\PY{p}{)}\PY{p}{)}\PY{p}{:}
             \PY{k}{if} \PY{n}{StatusStacked}\PY{p}{[}\PY{l+s+s1}{\PYZsq{}}\PY{l+s+s1}{Billed}\PY{l+s+s1}{\PYZsq{}}\PY{p}{]}\PY{p}{[}\PY{n}{i}\PY{p}{]} \PY{o}{\PYZlt{}} \PY{l+m+mi}{1} \PY{o+ow}{and} \PY{n}{StatusStacked}\PY{p}{[}\PY{l+s+s1}{\PYZsq{}}\PY{l+s+s1}{Next}\PY{l+s+s1}{\PYZsq{}}\PY{p}{]}\PY{p}{[}\PY{n}{i}\PY{p}{]}\PY{o}{\PYZgt{}}\PY{l+m+mi}{0}\PY{p}{:}
                 \PY{n}{ProblemCountV\PYZus{}Zero} \PY{o}{=} \PY{n}{ProblemCountV\PYZus{}Zero} \PY{o}{+} \PY{l+m+mi}{1}        
                 
         
         \PY{n+nb}{print}\PY{p}{(}\PY{l+s+s2}{\PYZdq{}}\PY{l+s+s2}{Versus \PYZhy{}1: }\PY{l+s+s2}{\PYZdq{}}\PY{p}{,} \PY{n}{ProblemCountV\PYZus{}Minus1}\PY{p}{,}\PY{l+s+s2}{\PYZdq{}}\PY{l+s+s2}{Versus 0: }\PY{l+s+s2}{\PYZdq{}}\PY{p}{,} \PY{n}{ProblemCountV\PYZus{}Zero}\PY{p}{)}
\end{Verbatim}


    \begin{Verbatim}[commandchars=\\\{\}]
Versus -1:  4702 Versus 0:  1769

    \end{Verbatim}

    As mentioned in the "Suspicious Values" section of this analysis: The
data points above show that there are instances where the Payment Status
implies late payment but the amount billed was zero or less. This seems
nonsensical. There are 1,769 such cases (out of 150,000 possible
payments: 30,000 accounts and 5 months of validatable data) if measured
against payment statuses defined as delinquent by the data providers,
and 4,702 if we consider Payment Status of zero as delinquent.

    \subsection{Explore Attributes and
Class}\label{explore-attributes-and-class}

In this section we review the various relationships between our various
features and there relationship to our target class (default =1). This
will allow us to understand certain features or potentially identify new
features for use in our model.

We began by exploring the relationships between the available
demographic data to the default class.

    \subparagraph{\texorpdfstring{\emph{Customer Demographics and
Default}}{Customer Demographics and Default}}\label{customer-demographics-and-default}

    \begin{Verbatim}[commandchars=\\\{\}]
{\color{incolor}In [{\color{incolor}28}]:} \PY{c+c1}{\PYZsh{}do some transformations}
         \PY{c+c1}{\PYZsh{}convert any non\PYZhy{}identified education categories to \PYZsq{}OTHER\PYZsq{}}
         \PY{n}{df}\PY{p}{[}\PY{l+s+s1}{\PYZsq{}}\PY{l+s+s1}{EDUCATION}\PY{l+s+s1}{\PYZsq{}}\PY{p}{]} \PY{o}{=} \PY{n}{df}\PY{p}{[}\PY{l+s+s1}{\PYZsq{}}\PY{l+s+s1}{EDUCATION}\PY{l+s+s1}{\PYZsq{}}\PY{p}{]}\PY{o}{.}\PY{n}{replace}\PY{p}{(}\PY{n}{to\PYZus{}replace}\PY{o}{=}\PY{p}{(}\PY{l+m+mi}{0}\PY{p}{,}\PY{l+m+mi}{5}\PY{p}{,}\PY{l+m+mi}{6}\PY{p}{)}\PY{p}{,}\PY{n}{value}\PY{o}{=}\PY{l+m+mi}{4}\PY{p}{)}
         
         \PY{c+c1}{\PYZsh{}convert any non\PYZhy{}identified marriage categories to \PYZsq{}OTHER\PYZsq{}}
         \PY{n}{df}\PY{p}{[}\PY{l+s+s1}{\PYZsq{}}\PY{l+s+s1}{MARRIAGE}\PY{l+s+s1}{\PYZsq{}}\PY{p}{]} \PY{o}{=} \PY{n}{df}\PY{p}{[}\PY{l+s+s1}{\PYZsq{}}\PY{l+s+s1}{MARRIAGE}\PY{l+s+s1}{\PYZsq{}}\PY{p}{]}\PY{o}{.}\PY{n}{replace}\PY{p}{(}\PY{n}{to\PYZus{}replace}\PY{o}{=}\PY{p}{(}\PY{l+m+mi}{0}\PY{p}{)}\PY{p}{,}\PY{n}{value}\PY{o}{=}\PY{l+m+mi}{3}\PY{p}{)}
         
         \PY{c+c1}{\PYZsh{}calculate the credit usage values}
         \PY{n}{df}\PY{p}{[}\PY{l+s+s1}{\PYZsq{}}\PY{l+s+s1}{USAGE\PYZus{}1}\PY{l+s+s1}{\PYZsq{}}\PY{p}{]} \PY{o}{=} \PY{n}{df}\PY{p}{[}\PY{l+s+s1}{\PYZsq{}}\PY{l+s+s1}{BILL\PYZus{}AMT1}\PY{l+s+s1}{\PYZsq{}}\PY{p}{]}\PY{o}{/}\PY{n}{df}\PY{p}{[}\PY{l+s+s1}{\PYZsq{}}\PY{l+s+s1}{LIMIT\PYZus{}BAL}\PY{l+s+s1}{\PYZsq{}}\PY{p}{]}
         \PY{n}{df}\PY{p}{[}\PY{l+s+s1}{\PYZsq{}}\PY{l+s+s1}{USAGE\PYZus{}2}\PY{l+s+s1}{\PYZsq{}}\PY{p}{]} \PY{o}{=} \PY{n}{df}\PY{p}{[}\PY{l+s+s1}{\PYZsq{}}\PY{l+s+s1}{BILL\PYZus{}AMT2}\PY{l+s+s1}{\PYZsq{}}\PY{p}{]}\PY{o}{/}\PY{n}{df}\PY{p}{[}\PY{l+s+s1}{\PYZsq{}}\PY{l+s+s1}{LIMIT\PYZus{}BAL}\PY{l+s+s1}{\PYZsq{}}\PY{p}{]}
         \PY{n}{df}\PY{p}{[}\PY{l+s+s1}{\PYZsq{}}\PY{l+s+s1}{USAGE\PYZus{}3}\PY{l+s+s1}{\PYZsq{}}\PY{p}{]} \PY{o}{=} \PY{n}{df}\PY{p}{[}\PY{l+s+s1}{\PYZsq{}}\PY{l+s+s1}{BILL\PYZus{}AMT3}\PY{l+s+s1}{\PYZsq{}}\PY{p}{]}\PY{o}{/}\PY{n}{df}\PY{p}{[}\PY{l+s+s1}{\PYZsq{}}\PY{l+s+s1}{LIMIT\PYZus{}BAL}\PY{l+s+s1}{\PYZsq{}}\PY{p}{]}
         \PY{n}{df}\PY{p}{[}\PY{l+s+s1}{\PYZsq{}}\PY{l+s+s1}{USAGE\PYZus{}4}\PY{l+s+s1}{\PYZsq{}}\PY{p}{]} \PY{o}{=} \PY{n}{df}\PY{p}{[}\PY{l+s+s1}{\PYZsq{}}\PY{l+s+s1}{BILL\PYZus{}AMT4}\PY{l+s+s1}{\PYZsq{}}\PY{p}{]}\PY{o}{/}\PY{n}{df}\PY{p}{[}\PY{l+s+s1}{\PYZsq{}}\PY{l+s+s1}{LIMIT\PYZus{}BAL}\PY{l+s+s1}{\PYZsq{}}\PY{p}{]}
         \PY{n}{df}\PY{p}{[}\PY{l+s+s1}{\PYZsq{}}\PY{l+s+s1}{USAGE\PYZus{}5}\PY{l+s+s1}{\PYZsq{}}\PY{p}{]} \PY{o}{=} \PY{n}{df}\PY{p}{[}\PY{l+s+s1}{\PYZsq{}}\PY{l+s+s1}{BILL\PYZus{}AMT5}\PY{l+s+s1}{\PYZsq{}}\PY{p}{]}\PY{o}{/}\PY{n}{df}\PY{p}{[}\PY{l+s+s1}{\PYZsq{}}\PY{l+s+s1}{LIMIT\PYZus{}BAL}\PY{l+s+s1}{\PYZsq{}}\PY{p}{]}
         \PY{n}{df}\PY{p}{[}\PY{l+s+s1}{\PYZsq{}}\PY{l+s+s1}{USAGE\PYZus{}6}\PY{l+s+s1}{\PYZsq{}}\PY{p}{]} \PY{o}{=} \PY{n}{df}\PY{p}{[}\PY{l+s+s1}{\PYZsq{}}\PY{l+s+s1}{BILL\PYZus{}AMT6}\PY{l+s+s1}{\PYZsq{}}\PY{p}{]}\PY{o}{/}\PY{n}{df}\PY{p}{[}\PY{l+s+s1}{\PYZsq{}}\PY{l+s+s1}{LIMIT\PYZus{}BAL}\PY{l+s+s1}{\PYZsq{}}\PY{p}{]}
         
         \PY{n}{payments} \PY{o}{=} \PY{p}{[}\PY{l+s+s1}{\PYZsq{}}\PY{l+s+s1}{PAY\PYZus{}0}\PY{l+s+s1}{\PYZsq{}}\PY{p}{,}\PY{l+s+s1}{\PYZsq{}}\PY{l+s+s1}{PAY\PYZus{}2}\PY{l+s+s1}{\PYZsq{}}\PY{p}{,}\PY{l+s+s1}{\PYZsq{}}\PY{l+s+s1}{PAY\PYZus{}3}\PY{l+s+s1}{\PYZsq{}}\PY{p}{,}\PY{l+s+s1}{\PYZsq{}}\PY{l+s+s1}{PAY\PYZus{}4}\PY{l+s+s1}{\PYZsq{}}\PY{p}{,}\PY{l+s+s1}{\PYZsq{}}\PY{l+s+s1}{PAY\PYZus{}5}\PY{l+s+s1}{\PYZsq{}}\PY{p}{,}\PY{l+s+s1}{\PYZsq{}}\PY{l+s+s1}{PAY\PYZus{}6}\PY{l+s+s1}{\PYZsq{}}\PY{p}{]}
         \PY{n}{df}\PY{p}{[}\PY{l+s+s1}{\PYZsq{}}\PY{l+s+s1}{TotalMonthsLate}\PY{l+s+s1}{\PYZsq{}}\PY{p}{]} \PY{o}{=} \PY{n}{df}\PY{p}{[}\PY{n}{payments}\PY{p}{]}\PY{o}{.}\PY{n}{sum}\PY{p}{(}\PY{n}{axis}\PY{o}{=}\PY{l+m+mi}{1}\PY{p}{)}
         \PY{c+c1}{\PYZsh{}transform continuous variables as they each have a mostly exponential distribution}
         \PY{n}{df}\PY{p}{[}\PY{n}{continuous\PYZus{}features}\PY{p}{]} \PY{o}{=} \PY{n}{df}\PY{p}{[}\PY{n}{continuous\PYZus{}features}\PY{p}{]}\PY{o}{.}\PY{n}{replace}\PY{p}{(}\PY{n}{to\PYZus{}replace}\PY{o}{=}\PY{l+m+mi}{0}\PY{p}{,}\PY{n}{value}\PY{o}{=}\PY{n}{np}\PY{o}{.}\PY{n}{nan}\PY{p}{)}\PY{o}{.}\PY{n}{apply}\PY{p}{(}\PY{n}{np}\PY{o}{.}\PY{n}{log}\PY{p}{)}
\end{Verbatim}


    \begin{Verbatim}[commandchars=\\\{\}]
{\color{incolor}In [{\color{incolor}29}]:} \PY{c+c1}{\PYZsh{} this python magics will allow plot to be embedded into the notebook}
         \PY{o}{\PYZpc{}}\PY{k}{matplotlib} inline
         \PY{c+c1}{\PYZsh{} cross tabs provide a quick view of the relationships between characteristics of the}
         \PY{c+c1}{\PYZsh{}borrower \PYZam{} our target}
         \PY{n}{plotVar} \PY{o}{=} \PY{p}{[}\PY{l+s+s1}{\PYZsq{}}\PY{l+s+s1}{SEX}\PY{l+s+s1}{\PYZsq{}}\PY{p}{,}\PY{l+s+s1}{\PYZsq{}}\PY{l+s+s1}{EDUCATION}\PY{l+s+s1}{\PYZsq{}}\PY{p}{,}\PY{l+s+s1}{\PYZsq{}}\PY{l+s+s1}{MARRIAGE}\PY{l+s+s1}{\PYZsq{}}\PY{p}{,}\PY{l+s+s1}{\PYZsq{}}\PY{l+s+s1}{AGE}\PY{l+s+s1}{\PYZsq{}}\PY{p}{]}
         
         \PY{n}{fig}\PY{p}{,} \PY{n}{axes} \PY{o}{=} \PY{n}{plt}\PY{o}{.}\PY{n}{subplots}\PY{p}{(}\PY{n}{nrows}\PY{o}{=}\PY{n+nb}{len}\PY{p}{(}\PY{n}{plotVar}\PY{p}{)}\PY{p}{,} \PY{n}{ncols}\PY{o}{=}\PY{l+m+mi}{2}\PY{p}{,} \PY{n}{figsize}\PY{o}{=}\PY{p}{(}\PY{l+m+mi}{15}\PY{p}{,} \PY{l+m+mi}{25}\PY{p}{)}\PY{p}{)}
         
         \PY{k}{for} \PY{n}{fi}\PY{p}{,}\PY{n}{feature} \PY{o+ow}{in} \PY{n+nb}{enumerate}\PY{p}{(}\PY{n}{plotVar}\PY{p}{)}\PY{p}{:}
             \PY{n}{Counts} \PY{o}{=} \PY{n}{pd}\PY{o}{.}\PY{n}{crosstab}\PY{p}{(}\PY{n}{df}\PY{p}{[}\PY{n}{feature}\PY{p}{]}\PY{p}{,}\PY{n}{df}\PY{o}{.}\PY{n}{default}\PY{o}{.}\PY{n}{astype}\PY{p}{(}\PY{n+nb}{bool}\PY{p}{)}\PY{p}{)}
             \PY{n}{Counts}\PY{o}{.}\PY{n}{plot}\PY{p}{(}\PY{n}{kind}\PY{o}{=}\PY{l+s+s1}{\PYZsq{}}\PY{l+s+s1}{bar}\PY{l+s+s1}{\PYZsq{}}\PY{p}{,} \PY{n}{stacked}\PY{o}{=}\PY{k+kc}{True}\PY{p}{,} \PY{n}{ax}\PY{o}{=}\PY{n}{axes}\PY{p}{[}\PY{n}{fi}\PY{p}{,}\PY{l+m+mi}{0}\PY{p}{]}\PY{p}{)}
             
             \PY{n}{Rate} \PY{o}{=} \PY{n}{Counts}\PY{o}{.}\PY{n}{div}\PY{p}{(}\PY{n}{Counts}\PY{o}{.}\PY{n}{sum}\PY{p}{(}\PY{l+m+mi}{1}\PY{p}{)}\PY{o}{.}\PY{n}{astype}\PY{p}{(}\PY{n+nb}{float}\PY{p}{)}\PY{p}{,}\PY{n}{axis}\PY{o}{=}\PY{l+m+mi}{0}\PY{p}{)}
             \PY{n}{Rate}\PY{o}{.}\PY{n}{plot}\PY{p}{(}\PY{n}{kind}\PY{o}{=}\PY{l+s+s1}{\PYZsq{}}\PY{l+s+s1}{barh}\PY{l+s+s1}{\PYZsq{}}\PY{p}{,} \PY{n}{stacked}\PY{o}{=}\PY{k+kc}{True}\PY{p}{,} \PY{n}{ax}\PY{o}{=}\PY{n}{axes}\PY{p}{[}\PY{n}{fi}\PY{p}{,}\PY{l+m+mi}{1}\PY{p}{]}\PY{p}{)}
             
         \PY{n}{plt}\PY{o}{.}\PY{n}{show}\PY{p}{(}\PY{p}{)}
\end{Verbatim}


    \begin{center}
    \adjustimage{max size={0.9\linewidth}{0.9\paperheight}}{panek_tanyuk_wall_davieau_lab1_files/panek_tanyuk_wall_davieau_lab1_59_0.png}
    \end{center}
    { \hspace*{\fill} \\}
    
    After reviewing the features from above there are a couple relationships
that stand out when comparing the demographic features and the
proportion of defaults for our sample. The review of this data below
simply addresses the variable relationships to the default class. It
does not address the potential ethical or legal concerns of using a
customer's demographic information to influence a bank's decision on
credit limits or interest rates.

Sex - There does not appear to be a noticeable difference in the default
rates by gender based on a visual inspections.

Education - There does seem to pattern between the amount of education
people receive and the default rates. Based on the visual inspection of
the above data there is some indication that people with higher levels
of education default at lower rates.

Marriage - There does not seem to be any obvious discernable trends
based on Marital Status

Age: - Based on the age variables there is potentially some indications
that people in there early 20's are more succeptible to defaulting,
however this variable may need to be bucketed differently to better
interpret the succeptible age groups. We will cover this in our 'New
Features' section.

    \subparagraph{\texorpdfstring{\emph{Payment History and
Default}}{Payment History and Default}}\label{payment-history-and-default}

After looking through the customers demographic history we wanted to
explore the relationships between the customers bill payment history to
identify any patterns that emerge.

We begin with the relationship between prior payment status (late
payments vs ontime)

    \begin{Verbatim}[commandchars=\\\{\}]
{\color{incolor}In [{\color{incolor}30}]:} \PY{n}{plotVar} \PY{o}{=} \PY{p}{[}\PY{l+s+s1}{\PYZsq{}}\PY{l+s+s1}{PAY\PYZus{}0}\PY{l+s+s1}{\PYZsq{}}\PY{p}{,}\PY{l+s+s1}{\PYZsq{}}\PY{l+s+s1}{PAY\PYZus{}2}\PY{l+s+s1}{\PYZsq{}}\PY{p}{,}\PY{l+s+s1}{\PYZsq{}}\PY{l+s+s1}{PAY\PYZus{}3}\PY{l+s+s1}{\PYZsq{}}\PY{p}{,}\PY{l+s+s1}{\PYZsq{}}\PY{l+s+s1}{PAY\PYZus{}4}\PY{l+s+s1}{\PYZsq{}}\PY{p}{,}\PY{l+s+s1}{\PYZsq{}}\PY{l+s+s1}{PAY\PYZus{}5}\PY{l+s+s1}{\PYZsq{}}\PY{p}{,}\PY{l+s+s1}{\PYZsq{}}\PY{l+s+s1}{PAY\PYZus{}6}\PY{l+s+s1}{\PYZsq{}}\PY{p}{,}\PY{l+s+s1}{\PYZsq{}}\PY{l+s+s1}{default}\PY{l+s+s1}{\PYZsq{}}\PY{p}{]}
         \PY{n}{plotDF} \PY{o}{=} \PY{n}{df}\PY{p}{[}\PY{n}{plotVar}\PY{p}{]}
         
         \PY{n}{meanDF} \PY{o}{=} \PY{n}{plotDF}\PY{o}{.}\PY{n}{groupby}\PY{p}{(}\PY{n}{by}\PY{o}{=}\PY{l+s+s1}{\PYZsq{}}\PY{l+s+s1}{default}\PY{l+s+s1}{\PYZsq{}}\PY{p}{)}\PY{o}{.}\PY{n}{mean}\PY{p}{(}\PY{p}{)}\PY{o}{.}\PY{n}{T}\PY{o}{.}\PY{n}{reset\PYZus{}index}\PY{p}{(}\PY{p}{)}
         \PY{n}{meanDF}\PY{p}{[}\PY{l+s+s1}{\PYZsq{}}\PY{l+s+s1}{Month}\PY{l+s+s1}{\PYZsq{}}\PY{p}{]} \PY{o}{=} \PY{p}{[}\PY{l+m+mi}{9}\PY{p}{,}\PY{l+m+mi}{8}\PY{p}{,}\PY{l+m+mi}{7}\PY{p}{,}\PY{l+m+mi}{6}\PY{p}{,}\PY{l+m+mi}{5}\PY{p}{,}\PY{l+m+mi}{4}\PY{p}{]}
         \PY{n}{meanDF}\PY{o}{.}\PY{n}{columns} \PY{o}{=} \PY{p}{[}\PY{l+s+s1}{\PYZsq{}}\PY{l+s+s1}{PreviousMonths}\PY{l+s+s1}{\PYZsq{}}\PY{p}{,} \PY{l+s+s1}{\PYZsq{}}\PY{l+s+s1}{False}\PY{l+s+s1}{\PYZsq{}}\PY{p}{,} \PY{l+s+s1}{\PYZsq{}}\PY{l+s+s1}{True}\PY{l+s+s1}{\PYZsq{}}\PY{p}{,} \PY{l+s+s1}{\PYZsq{}}\PY{l+s+s1}{Month}\PY{l+s+s1}{\PYZsq{}}\PY{p}{]}
         \PY{n}{meanDF} \PY{o}{=} \PY{n}{meanDF}\PY{o}{.}\PY{n}{sort\PYZus{}values}\PY{p}{(}\PY{n}{by}\PY{o}{=}\PY{p}{[}\PY{l+s+s1}{\PYZsq{}}\PY{l+s+s1}{Month}\PY{l+s+s1}{\PYZsq{}}\PY{p}{]}\PY{p}{,} \PY{n}{ascending}\PY{o}{=}\PY{k+kc}{True}\PY{p}{)}
         
         \PY{n}{meanDF}\PY{o}{.}\PY{n}{plot}\PY{p}{(}\PY{n}{x}\PY{o}{=}\PY{l+s+s1}{\PYZsq{}}\PY{l+s+s1}{PreviousMonths}\PY{l+s+s1}{\PYZsq{}}\PY{p}{,} \PY{n}{y}\PY{o}{=}\PY{p}{[}\PY{l+s+s1}{\PYZsq{}}\PY{l+s+s1}{False}\PY{l+s+s1}{\PYZsq{}}\PY{p}{,} \PY{l+s+s1}{\PYZsq{}}\PY{l+s+s1}{True}\PY{l+s+s1}{\PYZsq{}}\PY{p}{]}\PY{p}{,} \PY{n}{figsize}\PY{o}{=}\PY{p}{(}\PY{l+m+mi}{10}\PY{p}{,}\PY{l+m+mi}{5}\PY{p}{)}\PY{p}{,} \PY{n}{grid}\PY{o}{=}\PY{k+kc}{True}\PY{p}{,} \PY{n}{lw} \PY{o}{=} \PY{l+m+mi}{2}\PY{p}{)}
         \PY{n}{plt}\PY{o}{.}\PY{n}{title}\PY{p}{(}\PY{l+s+s1}{\PYZsq{}}\PY{l+s+s1}{Average Months Late in Last 6 Month (True = Default | False = Paid)}\PY{l+s+s1}{\PYZsq{}}\PY{p}{)}
         \PY{n}{plt}\PY{o}{.}\PY{n}{xlabel}\PY{p}{(}\PY{l+s+s1}{\PYZsq{}}\PY{l+s+s1}{Previous Months}\PY{l+s+s1}{\PYZsq{}}\PY{p}{)}
         \PY{n}{plt}\PY{o}{.}\PY{n}{ylabel}\PY{p}{(}\PY{l+s+s1}{\PYZsq{}}\PY{l+s+s1}{Average Months Behind Payment Due Date}\PY{l+s+s1}{\PYZsq{}}\PY{p}{)}
\end{Verbatim}


\begin{Verbatim}[commandchars=\\\{\}]
{\color{outcolor}Out[{\color{outcolor}30}]:} Text(0,0.5,'Average Months Behind Payment Due Date')
\end{Verbatim}
            
    \begin{center}
    \adjustimage{max size={0.9\linewidth}{0.9\paperheight}}{panek_tanyuk_wall_davieau_lab1_files/panek_tanyuk_wall_davieau_lab1_62_1.png}
    \end{center}
    { \hspace*{\fill} \\}
    
    From above the customers that defaulted paid there previous bill between
.2 to .7 months late on average. Additionally, it appears that as we
move closer to the month we are predicting(left to right), the
timeliness of the bill payments seems to be getting worse for both
groups.

Logically we interpret this to support the notion that people who have
trouble paying there previous bills will likely have similar
difficulties paying in the future.

This lends to the potential for a new features that better captures the
timeliness of previous payments as well as the status of the last
payment (paid or still outstanding).

In order to understand why, we will look at the average of the log
transformed bill amounts to see if there is any identifiable pattern.

    \begin{Verbatim}[commandchars=\\\{\}]
{\color{incolor}In [{\color{incolor}31}]:} \PY{n}{plotVar} \PY{o}{=} \PY{p}{[}\PY{l+s+s1}{\PYZsq{}}\PY{l+s+s1}{BILL\PYZus{}AMT1}\PY{l+s+s1}{\PYZsq{}}\PY{p}{,} \PY{l+s+s1}{\PYZsq{}}\PY{l+s+s1}{BILL\PYZus{}AMT2}\PY{l+s+s1}{\PYZsq{}}\PY{p}{,}\PY{l+s+s1}{\PYZsq{}}\PY{l+s+s1}{BILL\PYZus{}AMT3}\PY{l+s+s1}{\PYZsq{}}\PY{p}{,}\PY{l+s+s1}{\PYZsq{}}\PY{l+s+s1}{BILL\PYZus{}AMT4}\PY{l+s+s1}{\PYZsq{}}\PY{p}{,} \PY{l+s+s1}{\PYZsq{}}\PY{l+s+s1}{BILL\PYZus{}AMT5}\PY{l+s+s1}{\PYZsq{}}\PY{p}{,} \PY{l+s+s1}{\PYZsq{}}\PY{l+s+s1}{BILL\PYZus{}AMT6}\PY{l+s+s1}{\PYZsq{}}\PY{p}{,}\PY{l+s+s1}{\PYZsq{}}\PY{l+s+s1}{default}\PY{l+s+s1}{\PYZsq{}}\PY{p}{]}
         \PY{n}{plotDF} \PY{o}{=} \PY{n}{df}\PY{p}{[}\PY{n}{plotVar}\PY{p}{]}
         
         \PY{n}{meanDF} \PY{o}{=} \PY{n}{plotDF}\PY{o}{.}\PY{n}{groupby}\PY{p}{(}\PY{n}{by}\PY{o}{=}\PY{l+s+s1}{\PYZsq{}}\PY{l+s+s1}{default}\PY{l+s+s1}{\PYZsq{}}\PY{p}{)}\PY{o}{.}\PY{n}{mean}\PY{p}{(}\PY{p}{)}\PY{o}{.}\PY{n}{T}\PY{o}{.}\PY{n}{reset\PYZus{}index}\PY{p}{(}\PY{p}{)}
         \PY{n}{meanDF}\PY{p}{[}\PY{l+s+s1}{\PYZsq{}}\PY{l+s+s1}{Month}\PY{l+s+s1}{\PYZsq{}}\PY{p}{]} \PY{o}{=} \PY{p}{[}\PY{l+m+mi}{9}\PY{p}{,}\PY{l+m+mi}{8}\PY{p}{,}\PY{l+m+mi}{7}\PY{p}{,}\PY{l+m+mi}{6}\PY{p}{,}\PY{l+m+mi}{5}\PY{p}{,}\PY{l+m+mi}{4}\PY{p}{]}
         \PY{n}{meanDF}\PY{o}{.}\PY{n}{columns} \PY{o}{=} \PY{p}{[}\PY{l+s+s1}{\PYZsq{}}\PY{l+s+s1}{PreviousMonths}\PY{l+s+s1}{\PYZsq{}}\PY{p}{,} \PY{l+s+s1}{\PYZsq{}}\PY{l+s+s1}{False}\PY{l+s+s1}{\PYZsq{}}\PY{p}{,} \PY{l+s+s1}{\PYZsq{}}\PY{l+s+s1}{True}\PY{l+s+s1}{\PYZsq{}}\PY{p}{,} \PY{l+s+s1}{\PYZsq{}}\PY{l+s+s1}{Month}\PY{l+s+s1}{\PYZsq{}}\PY{p}{]}
         \PY{n}{meanDF} \PY{o}{=} \PY{n}{meanDF}\PY{o}{.}\PY{n}{sort\PYZus{}values}\PY{p}{(}\PY{n}{by}\PY{o}{=}\PY{p}{[}\PY{l+s+s1}{\PYZsq{}}\PY{l+s+s1}{Month}\PY{l+s+s1}{\PYZsq{}}\PY{p}{]}\PY{p}{,} \PY{n}{ascending}\PY{o}{=}\PY{k+kc}{True}\PY{p}{)}
         
         \PY{n}{meanDF}\PY{o}{.}\PY{n}{plot}\PY{p}{(}\PY{n}{x}\PY{o}{=}\PY{l+s+s1}{\PYZsq{}}\PY{l+s+s1}{PreviousMonths}\PY{l+s+s1}{\PYZsq{}}\PY{p}{,} \PY{n}{y}\PY{o}{=}\PY{p}{[}\PY{l+s+s1}{\PYZsq{}}\PY{l+s+s1}{False}\PY{l+s+s1}{\PYZsq{}}\PY{p}{,} \PY{l+s+s1}{\PYZsq{}}\PY{l+s+s1}{True}\PY{l+s+s1}{\PYZsq{}}\PY{p}{]}\PY{p}{,} \PY{n}{figsize}\PY{o}{=}\PY{p}{(}\PY{l+m+mi}{10}\PY{p}{,}\PY{l+m+mi}{5}\PY{p}{)}\PY{p}{,} \PY{n}{grid}\PY{o}{=}\PY{k+kc}{True}\PY{p}{,} \PY{n}{lw} \PY{o}{=} \PY{l+m+mi}{2}\PY{p}{)}
         \PY{n}{plt}\PY{o}{.}\PY{n}{title}\PY{p}{(}\PY{l+s+s1}{\PYZsq{}}\PY{l+s+s1}{Average Log(Bill Amt) in Last 6 Month (True = Default | False = Paid)}\PY{l+s+s1}{\PYZsq{}}\PY{p}{)}
         \PY{n}{plt}\PY{o}{.}\PY{n}{xlabel}\PY{p}{(}\PY{l+s+s1}{\PYZsq{}}\PY{l+s+s1}{Previous Months}\PY{l+s+s1}{\PYZsq{}}\PY{p}{)}
         \PY{n}{plt}\PY{o}{.}\PY{n}{ylabel}\PY{p}{(}\PY{l+s+s1}{\PYZsq{}}\PY{l+s+s1}{Average Log(Bill Amt)}\PY{l+s+s1}{\PYZsq{}}\PY{p}{)}
\end{Verbatim}


\begin{Verbatim}[commandchars=\\\{\}]
{\color{outcolor}Out[{\color{outcolor}31}]:} Text(0,0.5,'Average Log(Bill Amt)')
\end{Verbatim}
            
    \begin{center}
    \adjustimage{max size={0.9\linewidth}{0.9\paperheight}}{panek_tanyuk_wall_davieau_lab1_files/panek_tanyuk_wall_davieau_lab1_64_1.png}
    \end{center}
    { \hspace*{\fill} \\}
    
    Interestingly, when looking at the bill amount from the previous months
there does not seem to be a considerable difference between the median
bill amount between customers that default and those that don't.

Based on this it would seem that the actual bill amount by itself will
offer little insight into the likelihood of default. This is perhaps an
indication that it should be considered relative to the total available
credit and the prior payments.

Next we look deeper that log transformed payment amounts.

    \begin{Verbatim}[commandchars=\\\{\}]
{\color{incolor}In [{\color{incolor}32}]:} \PY{n}{plotVar} \PY{o}{=} \PY{p}{[}\PY{l+s+s1}{\PYZsq{}}\PY{l+s+s1}{PAY\PYZus{}AMT1}\PY{l+s+s1}{\PYZsq{}}\PY{p}{,} \PY{l+s+s1}{\PYZsq{}}\PY{l+s+s1}{PAY\PYZus{}AMT2}\PY{l+s+s1}{\PYZsq{}}\PY{p}{,}\PY{l+s+s1}{\PYZsq{}}\PY{l+s+s1}{PAY\PYZus{}AMT3}\PY{l+s+s1}{\PYZsq{}}\PY{p}{,}\PY{l+s+s1}{\PYZsq{}}\PY{l+s+s1}{PAY\PYZus{}AMT4}\PY{l+s+s1}{\PYZsq{}}\PY{p}{,} \PY{l+s+s1}{\PYZsq{}}\PY{l+s+s1}{PAY\PYZus{}AMT5}\PY{l+s+s1}{\PYZsq{}}\PY{p}{,} \PY{l+s+s1}{\PYZsq{}}\PY{l+s+s1}{PAY\PYZus{}AMT6}\PY{l+s+s1}{\PYZsq{}}\PY{p}{,}\PY{l+s+s1}{\PYZsq{}}\PY{l+s+s1}{default}\PY{l+s+s1}{\PYZsq{}}\PY{p}{]}
         \PY{n}{plotDF} \PY{o}{=} \PY{n}{df}\PY{p}{[}\PY{n}{plotVar}\PY{p}{]}
         
         \PY{n}{meanDF} \PY{o}{=} \PY{n}{plotDF}\PY{o}{.}\PY{n}{groupby}\PY{p}{(}\PY{n}{by}\PY{o}{=}\PY{l+s+s1}{\PYZsq{}}\PY{l+s+s1}{default}\PY{l+s+s1}{\PYZsq{}}\PY{p}{)}\PY{o}{.}\PY{n}{mean}\PY{p}{(}\PY{p}{)}\PY{o}{.}\PY{n}{T}\PY{o}{.}\PY{n}{reset\PYZus{}index}\PY{p}{(}\PY{p}{)}
         \PY{n}{meanDF}\PY{p}{[}\PY{l+s+s1}{\PYZsq{}}\PY{l+s+s1}{Month}\PY{l+s+s1}{\PYZsq{}}\PY{p}{]} \PY{o}{=} \PY{p}{[}\PY{l+m+mi}{9}\PY{p}{,}\PY{l+m+mi}{8}\PY{p}{,}\PY{l+m+mi}{7}\PY{p}{,}\PY{l+m+mi}{6}\PY{p}{,}\PY{l+m+mi}{5}\PY{p}{,}\PY{l+m+mi}{4}\PY{p}{]}
         \PY{n}{meanDF}\PY{o}{.}\PY{n}{columns} \PY{o}{=} \PY{p}{[}\PY{l+s+s1}{\PYZsq{}}\PY{l+s+s1}{PreviousMonths}\PY{l+s+s1}{\PYZsq{}}\PY{p}{,} \PY{l+s+s1}{\PYZsq{}}\PY{l+s+s1}{False}\PY{l+s+s1}{\PYZsq{}}\PY{p}{,} \PY{l+s+s1}{\PYZsq{}}\PY{l+s+s1}{True}\PY{l+s+s1}{\PYZsq{}}\PY{p}{,} \PY{l+s+s1}{\PYZsq{}}\PY{l+s+s1}{Month}\PY{l+s+s1}{\PYZsq{}}\PY{p}{]}
         \PY{n}{meanDF} \PY{o}{=} \PY{n}{meanDF}\PY{o}{.}\PY{n}{sort\PYZus{}values}\PY{p}{(}\PY{n}{by}\PY{o}{=}\PY{p}{[}\PY{l+s+s1}{\PYZsq{}}\PY{l+s+s1}{Month}\PY{l+s+s1}{\PYZsq{}}\PY{p}{]}\PY{p}{,} \PY{n}{ascending}\PY{o}{=}\PY{k+kc}{True}\PY{p}{)}
         
         \PY{n}{meanDF}\PY{o}{.}\PY{n}{plot}\PY{p}{(}\PY{n}{x}\PY{o}{=}\PY{l+s+s1}{\PYZsq{}}\PY{l+s+s1}{PreviousMonths}\PY{l+s+s1}{\PYZsq{}}\PY{p}{,} \PY{n}{y}\PY{o}{=}\PY{p}{[}\PY{l+s+s1}{\PYZsq{}}\PY{l+s+s1}{False}\PY{l+s+s1}{\PYZsq{}}\PY{p}{,} \PY{l+s+s1}{\PYZsq{}}\PY{l+s+s1}{True}\PY{l+s+s1}{\PYZsq{}}\PY{p}{]}\PY{p}{,} \PY{n}{figsize}\PY{o}{=}\PY{p}{(}\PY{l+m+mi}{10}\PY{p}{,}\PY{l+m+mi}{5}\PY{p}{)}\PY{p}{,} \PY{n}{grid}\PY{o}{=}\PY{k+kc}{True}\PY{p}{,} \PY{n}{lw} \PY{o}{=} \PY{l+m+mi}{2}\PY{p}{)}
         \PY{n}{plt}\PY{o}{.}\PY{n}{title}\PY{p}{(}\PY{l+s+s1}{\PYZsq{}}\PY{l+s+s1}{Average Log(Pay Amt) in Last 6 Month (True = Default | False = Paid)}\PY{l+s+s1}{\PYZsq{}}\PY{p}{)}
         \PY{n}{plt}\PY{o}{.}\PY{n}{xlabel}\PY{p}{(}\PY{l+s+s1}{\PYZsq{}}\PY{l+s+s1}{Previous Months}\PY{l+s+s1}{\PYZsq{}}\PY{p}{)}
         \PY{n}{plt}\PY{o}{.}\PY{n}{ylabel}\PY{p}{(}\PY{l+s+s1}{\PYZsq{}}\PY{l+s+s1}{Average Log(Pay Amt)}\PY{l+s+s1}{\PYZsq{}}\PY{p}{)}
\end{Verbatim}


\begin{Verbatim}[commandchars=\\\{\}]
{\color{outcolor}Out[{\color{outcolor}32}]:} Text(0,0.5,'Average Log(Pay Amt)')
\end{Verbatim}
            
    \begin{center}
    \adjustimage{max size={0.9\linewidth}{0.9\paperheight}}{panek_tanyuk_wall_davieau_lab1_files/panek_tanyuk_wall_davieau_lab1_66_1.png}
    \end{center}
    { \hspace*{\fill} \\}
    
    There does appear to be a difference in the median payment amount
between the customers that defaulted and those that did not.

It appears that customers that did not default make higher payments
against their credit on average than those that do default.

This leads us to question if customers that default are paying less, but
maintain a similar bill amount then they are likely using a higher
proportion of there credit. We will introduce a new feature that will
measure the customer's credit usage in the 'New Features' section.

    \subparagraph{\texorpdfstring{\emph{Summary}}{Summary}}\label{summary}

Based on the reviews of the features above it appears that we have
several features that have a relationship to our target class.
Additionally, we identified several limitations of our current features
and have decided to include a several new features that could
potentially improve our model.

    \subsection{New Features}\label{new-features}

Based on the above analysis we believe we idenfied several new features
that we will derive from our existing data. Below we explain the
variable and the code used to create it.

    \subparagraph{\texorpdfstring{\emph{Bucketed Age
Group}}{Bucketed Age Group}}\label{bucketed-age-group}

This will allow us to pick up some of the traits of different age groups
and life events. For example young adults (21-28) will likely have a
different financial situation than a retiree.

    \begin{Verbatim}[commandchars=\\\{\}]
{\color{incolor}In [{\color{incolor}33}]:} \PY{c+c1}{\PYZsh{} Creating Age Buckets}
         \PY{n}{df}\PY{p}{[}\PY{l+s+s1}{\PYZsq{}}\PY{l+s+s1}{Agegroup}\PY{l+s+s1}{\PYZsq{}}\PY{p}{]} \PY{o}{=} \PY{n}{df}\PY{p}{[}\PY{l+s+s1}{\PYZsq{}}\PY{l+s+s1}{AGE}\PY{l+s+s1}{\PYZsq{}}\PY{p}{]}\PY{o}{/}\PY{o}{/}\PY{l+m+mi}{10}
\end{Verbatim}


    \subparagraph{\texorpdfstring{\emph{Credit Usage
History}}{Credit Usage History}}\label{credit-usage-history}

Using the billed amounts relative to the total available credit for each
of the customers gives us information on how much available credit they
are using and carrying over month to month. This feature will allow us
to determine how high consistently high credit usage effects a
customer's likelihood of defaulting.

    \begin{Verbatim}[commandchars=\\\{\}]
{\color{incolor}In [{\color{incolor}34}]:} \PY{c+c1}{\PYZsh{}calculate the credit usage values}
         \PY{n}{df}\PY{p}{[}\PY{l+s+s1}{\PYZsq{}}\PY{l+s+s1}{USAGE\PYZus{}1}\PY{l+s+s1}{\PYZsq{}}\PY{p}{]} \PY{o}{=} \PY{n}{df}\PY{p}{[}\PY{l+s+s1}{\PYZsq{}}\PY{l+s+s1}{BILL\PYZus{}AMT1}\PY{l+s+s1}{\PYZsq{}}\PY{p}{]}\PY{o}{/}\PY{n}{df}\PY{p}{[}\PY{l+s+s1}{\PYZsq{}}\PY{l+s+s1}{LIMIT\PYZus{}BAL}\PY{l+s+s1}{\PYZsq{}}\PY{p}{]}
         \PY{n}{df}\PY{p}{[}\PY{l+s+s1}{\PYZsq{}}\PY{l+s+s1}{USAGE\PYZus{}2}\PY{l+s+s1}{\PYZsq{}}\PY{p}{]} \PY{o}{=} \PY{n}{df}\PY{p}{[}\PY{l+s+s1}{\PYZsq{}}\PY{l+s+s1}{BILL\PYZus{}AMT2}\PY{l+s+s1}{\PYZsq{}}\PY{p}{]}\PY{o}{/}\PY{n}{df}\PY{p}{[}\PY{l+s+s1}{\PYZsq{}}\PY{l+s+s1}{LIMIT\PYZus{}BAL}\PY{l+s+s1}{\PYZsq{}}\PY{p}{]}
         \PY{n}{df}\PY{p}{[}\PY{l+s+s1}{\PYZsq{}}\PY{l+s+s1}{USAGE\PYZus{}3}\PY{l+s+s1}{\PYZsq{}}\PY{p}{]} \PY{o}{=} \PY{n}{df}\PY{p}{[}\PY{l+s+s1}{\PYZsq{}}\PY{l+s+s1}{BILL\PYZus{}AMT3}\PY{l+s+s1}{\PYZsq{}}\PY{p}{]}\PY{o}{/}\PY{n}{df}\PY{p}{[}\PY{l+s+s1}{\PYZsq{}}\PY{l+s+s1}{LIMIT\PYZus{}BAL}\PY{l+s+s1}{\PYZsq{}}\PY{p}{]}
         \PY{n}{df}\PY{p}{[}\PY{l+s+s1}{\PYZsq{}}\PY{l+s+s1}{USAGE\PYZus{}4}\PY{l+s+s1}{\PYZsq{}}\PY{p}{]} \PY{o}{=} \PY{n}{df}\PY{p}{[}\PY{l+s+s1}{\PYZsq{}}\PY{l+s+s1}{BILL\PYZus{}AMT4}\PY{l+s+s1}{\PYZsq{}}\PY{p}{]}\PY{o}{/}\PY{n}{df}\PY{p}{[}\PY{l+s+s1}{\PYZsq{}}\PY{l+s+s1}{LIMIT\PYZus{}BAL}\PY{l+s+s1}{\PYZsq{}}\PY{p}{]}
         \PY{n}{df}\PY{p}{[}\PY{l+s+s1}{\PYZsq{}}\PY{l+s+s1}{USAGE\PYZus{}5}\PY{l+s+s1}{\PYZsq{}}\PY{p}{]} \PY{o}{=} \PY{n}{df}\PY{p}{[}\PY{l+s+s1}{\PYZsq{}}\PY{l+s+s1}{BILL\PYZus{}AMT5}\PY{l+s+s1}{\PYZsq{}}\PY{p}{]}\PY{o}{/}\PY{n}{df}\PY{p}{[}\PY{l+s+s1}{\PYZsq{}}\PY{l+s+s1}{LIMIT\PYZus{}BAL}\PY{l+s+s1}{\PYZsq{}}\PY{p}{]}
         \PY{n}{df}\PY{p}{[}\PY{l+s+s1}{\PYZsq{}}\PY{l+s+s1}{USAGE\PYZus{}6}\PY{l+s+s1}{\PYZsq{}}\PY{p}{]} \PY{o}{=} \PY{n}{df}\PY{p}{[}\PY{l+s+s1}{\PYZsq{}}\PY{l+s+s1}{BILL\PYZus{}AMT6}\PY{l+s+s1}{\PYZsq{}}\PY{p}{]}\PY{o}{/}\PY{n}{df}\PY{p}{[}\PY{l+s+s1}{\PYZsq{}}\PY{l+s+s1}{LIMIT\PYZus{}BAL}\PY{l+s+s1}{\PYZsq{}}\PY{p}{]}
\end{Verbatim}


    \subparagraph{\texorpdfstring{\emph{Late Payments
Total}}{Late Payments Total}}\label{late-payments-total}

How many times has a customer been late on a payment in the last six
months? We saw in the section above that a history of late payments does
appear to have a strong relationship with our target variable default.

\subparagraph{\texorpdfstring{\emph{Changes in Client
Behavior}}{Changes in Client Behavior}}\label{changes-in-client-behavior}

Trends in Billed Amounts or \% of Billed Amounts paid may prove to be
related to the propensity to default.

    \begin{Verbatim}[commandchars=\\\{\}]
{\color{incolor}In [{\color{incolor}35}]:} \PY{n}{transformVar} \PY{o}{=} \PY{p}{[}\PY{l+s+s1}{\PYZsq{}}\PY{l+s+s1}{PAY\PYZus{}0}\PY{l+s+s1}{\PYZsq{}}\PY{p}{,}\PY{l+s+s1}{\PYZsq{}}\PY{l+s+s1}{PAY\PYZus{}2}\PY{l+s+s1}{\PYZsq{}}\PY{p}{,}\PY{l+s+s1}{\PYZsq{}}\PY{l+s+s1}{PAY\PYZus{}3}\PY{l+s+s1}{\PYZsq{}}\PY{p}{,}\PY{l+s+s1}{\PYZsq{}}\PY{l+s+s1}{PAY\PYZus{}4}\PY{l+s+s1}{\PYZsq{}}\PY{p}{,}\PY{l+s+s1}{\PYZsq{}}\PY{l+s+s1}{PAY\PYZus{}5}\PY{l+s+s1}{\PYZsq{}}\PY{p}{,}\PY{l+s+s1}{\PYZsq{}}\PY{l+s+s1}{PAY\PYZus{}6}\PY{l+s+s1}{\PYZsq{}}\PY{p}{]}
         
         \PY{k}{for} \PY{n}{fi}\PY{p}{,}\PY{n}{feature} \PY{o+ow}{in} \PY{n+nb}{enumerate}\PY{p}{(}\PY{n}{transformVar}\PY{p}{)}\PY{p}{:}
             \PY{n}{df}\PY{p}{[}\PY{n}{feature}\PY{p}{]} \PY{o}{=} \PY{n}{pd}\PY{o}{.}\PY{n}{cut}\PY{p}{(}\PY{n}{df}\PY{p}{[}\PY{n}{feature}\PY{p}{]}\PY{p}{,} \PY{p}{[}\PY{o}{\PYZhy{}}\PY{l+m+mi}{3}\PY{p}{,}\PY{o}{\PYZhy{}}\PY{l+m+mi}{1}\PY{p}{,}\PY{l+m+mi}{8}\PY{p}{]}\PY{p}{,} \PY{l+m+mi}{2}\PY{p}{,} \PY{n}{labels}\PY{o}{=}\PY{p}{[}\PY{l+m+mi}{0}\PY{p}{,}\PY{l+m+mi}{1}\PY{p}{]}\PY{p}{)}\PY{o}{.}\PY{n}{astype}\PY{p}{(}\PY{n}{np}\PY{o}{.}\PY{n}{int64}\PY{p}{)}
         
         \PY{n}{df}\PY{p}{[}\PY{l+s+s1}{\PYZsq{}}\PY{l+s+s1}{TotalMonthsLate}\PY{l+s+s1}{\PYZsq{}}\PY{p}{]} \PY{o}{=} \PY{n}{df}\PY{p}{[}\PY{n}{transformVar}\PY{p}{]}\PY{o}{.}\PY{n}{sum}\PY{p}{(}\PY{n}{axis}\PY{o}{=}\PY{l+m+mi}{1}\PY{p}{)}
\end{Verbatim}


    \subsection{Exceptional Work}\label{exceptional-work}

    \begin{Verbatim}[commandchars=\\\{\}]
{\color{incolor}In [{\color{incolor}36}]:} \PY{n}{df} \PY{o}{=} \PY{n}{pd}\PY{o}{.}\PY{n}{read\PYZus{}csv}\PY{p}{(}\PY{l+s+s1}{\PYZsq{}}\PY{l+s+s1}{Input/DefaultCreditcardClients.csv}\PY{l+s+s1}{\PYZsq{}}\PY{p}{)}
         \PY{n}{df}\PY{o}{.}\PY{n}{rename}\PY{p}{(}\PY{n}{columns}\PY{o}{=}\PY{p}{\PYZob{}}\PY{l+s+s1}{\PYZsq{}}\PY{l+s+s1}{default payment next month}\PY{l+s+s1}{\PYZsq{}}\PY{p}{:}\PY{l+s+s1}{\PYZsq{}}\PY{l+s+s1}{default}\PY{l+s+s1}{\PYZsq{}}\PY{p}{\PYZcb{}}\PY{p}{,} \PY{n}{inplace}\PY{o}{=}\PY{k+kc}{True}\PY{p}{)}
         \PY{n}{df}\PY{o}{.}\PY{n}{index} \PY{o}{=} \PY{n}{df}\PY{o}{.}\PY{n}{ID}
         \PY{k}{if} \PY{l+s+s1}{\PYZsq{}}\PY{l+s+s1}{ID}\PY{l+s+s1}{\PYZsq{}} \PY{o+ow}{in} \PY{n}{df}\PY{p}{:}
             \PY{k}{del} \PY{n}{df}\PY{p}{[}\PY{l+s+s1}{\PYZsq{}}\PY{l+s+s1}{ID}\PY{l+s+s1}{\PYZsq{}}\PY{p}{]}
             
         \PY{n}{df}\PY{p}{[}\PY{l+s+s2}{\PYZdq{}}\PY{l+s+s2}{log\PYZus{}LIMIT\PYZus{}BAL}\PY{l+s+s2}{\PYZdq{}}\PY{p}{]}\PY{o}{=}\PY{n}{np}\PY{o}{.}\PY{n}{log}\PY{p}{(}\PY{n}{df}\PY{o}{.}\PY{n}{LIMIT\PYZus{}BAL}\PY{p}{)}
         \PY{n}{df}\PY{p}{[}\PY{l+s+s2}{\PYZdq{}}\PY{l+s+s2}{log\PYZus{}PAY\PYZus{}AMT1}\PY{l+s+s2}{\PYZdq{}}\PY{p}{]}\PY{o}{=}\PY{n}{np}\PY{o}{.}\PY{n}{log}\PY{p}{(}\PY{n}{df}\PY{o}{.}\PY{n}{PAY\PYZus{}AMT1}\PY{o}{+}\PY{l+m+mi}{1}\PY{p}{)}
         \PY{n}{df}\PY{p}{[}\PY{l+s+s2}{\PYZdq{}}\PY{l+s+s2}{log\PYZus{}PAY\PYZus{}AMT2}\PY{l+s+s2}{\PYZdq{}}\PY{p}{]}\PY{o}{=}\PY{n}{np}\PY{o}{.}\PY{n}{log}\PY{p}{(}\PY{n}{df}\PY{o}{.}\PY{n}{PAY\PYZus{}AMT2}\PY{o}{+}\PY{l+m+mi}{1}\PY{p}{)}
         \PY{n}{df}\PY{p}{[}\PY{l+s+s2}{\PYZdq{}}\PY{l+s+s2}{log\PYZus{}PAY\PYZus{}AMT3}\PY{l+s+s2}{\PYZdq{}}\PY{p}{]}\PY{o}{=}\PY{n}{np}\PY{o}{.}\PY{n}{log}\PY{p}{(}\PY{n}{df}\PY{o}{.}\PY{n}{PAY\PYZus{}AMT3}\PY{o}{+}\PY{l+m+mi}{1}\PY{p}{)}
         \PY{n}{df}\PY{p}{[}\PY{l+s+s2}{\PYZdq{}}\PY{l+s+s2}{log\PYZus{}PAY\PYZus{}AMT4}\PY{l+s+s2}{\PYZdq{}}\PY{p}{]}\PY{o}{=}\PY{n}{np}\PY{o}{.}\PY{n}{log}\PY{p}{(}\PY{n}{df}\PY{o}{.}\PY{n}{PAY\PYZus{}AMT4}\PY{o}{+}\PY{l+m+mi}{1}\PY{p}{)}
         \PY{n}{df}\PY{p}{[}\PY{l+s+s2}{\PYZdq{}}\PY{l+s+s2}{log\PYZus{}PAY\PYZus{}AMT5}\PY{l+s+s2}{\PYZdq{}}\PY{p}{]}\PY{o}{=}\PY{n}{np}\PY{o}{.}\PY{n}{log}\PY{p}{(}\PY{n}{df}\PY{o}{.}\PY{n}{PAY\PYZus{}AMT5}\PY{o}{+}\PY{l+m+mi}{1}\PY{p}{)}
         \PY{n}{df}\PY{p}{[}\PY{l+s+s2}{\PYZdq{}}\PY{l+s+s2}{log\PYZus{}PAY\PYZus{}AMT6}\PY{l+s+s2}{\PYZdq{}}\PY{p}{]}\PY{o}{=}\PY{n}{np}\PY{o}{.}\PY{n}{log}\PY{p}{(}\PY{n}{df}\PY{o}{.}\PY{n}{PAY\PYZus{}AMT6}\PY{o}{+}\PY{l+m+mi}{1}\PY{p}{)}
         
         \PY{k+kn}{from} \PY{n+nn}{sklearn}\PY{n+nn}{.}\PY{n+nn}{decomposition} \PY{k}{import} \PY{n}{PCA}
         \PY{n}{pca}\PY{o}{=}\PY{n}{PCA}\PY{p}{(}\PY{n}{n\PYZus{}components}\PY{o}{=}\PY{l+m+mi}{4}\PY{p}{)}
         \PY{n}{X}\PY{o}{=}\PY{n}{df}\PY{p}{[}\PY{p}{[}\PY{l+s+s1}{\PYZsq{}}\PY{l+s+s1}{log\PYZus{}LIMIT\PYZus{}BAL}\PY{l+s+s1}{\PYZsq{}}\PY{p}{,} \PY{l+s+s1}{\PYZsq{}}\PY{l+s+s1}{BILL\PYZus{}AMT1}\PY{l+s+s1}{\PYZsq{}}\PY{p}{,} \PY{l+s+s1}{\PYZsq{}}\PY{l+s+s1}{BILL\PYZus{}AMT2}\PY{l+s+s1}{\PYZsq{}}\PY{p}{,}\PY{l+s+s1}{\PYZsq{}}\PY{l+s+s1}{BILL\PYZus{}AMT3}\PY{l+s+s1}{\PYZsq{}}\PY{p}{,}\PY{l+s+s1}{\PYZsq{}}\PY{l+s+s1}{BILL\PYZus{}AMT4}\PY{l+s+s1}{\PYZsq{}}\PY{p}{,} \PY{l+s+s1}{\PYZsq{}}\PY{l+s+s1}{BILL\PYZus{}AMT5}\PY{l+s+s1}{\PYZsq{}}\PY{p}{,} \PY{l+s+s1}{\PYZsq{}}\PY{l+s+s1}{BILL\PYZus{}AMT6}\PY{l+s+s1}{\PYZsq{}}\PY{p}{,}
                         \PY{l+s+s1}{\PYZsq{}}\PY{l+s+s1}{log\PYZus{}PAY\PYZus{}AMT1}\PY{l+s+s1}{\PYZsq{}}\PY{p}{,}\PY{l+s+s1}{\PYZsq{}}\PY{l+s+s1}{log\PYZus{}PAY\PYZus{}AMT2}\PY{l+s+s1}{\PYZsq{}}\PY{p}{,} \PY{l+s+s1}{\PYZsq{}}\PY{l+s+s1}{log\PYZus{}PAY\PYZus{}AMT3}\PY{l+s+s1}{\PYZsq{}}\PY{p}{,} \PY{l+s+s1}{\PYZsq{}}\PY{l+s+s1}{log\PYZus{}PAY\PYZus{}AMT4}\PY{l+s+s1}{\PYZsq{}}\PY{p}{,} \PY{l+s+s1}{\PYZsq{}}\PY{l+s+s1}{log\PYZus{}PAY\PYZus{}AMT5}\PY{l+s+s1}{\PYZsq{}}\PY{p}{,}\PY{l+s+s1}{\PYZsq{}}\PY{l+s+s1}{log\PYZus{}PAY\PYZus{}AMT6}\PY{l+s+s1}{\PYZsq{}}\PY{p}{]}\PY{p}{]}
         \PY{n}{X\PYZus{}pca} \PY{o}{=} \PY{n}{pca}\PY{o}{.}\PY{n}{fit}\PY{p}{(}\PY{n}{X}\PY{p}{)}\PY{o}{.}\PY{n}{transform}\PY{p}{(}\PY{n}{X}\PY{p}{)}
         \PY{n+nb}{print} \PY{p}{(}\PY{l+s+s1}{\PYZsq{}}\PY{l+s+s1}{pca:}\PY{l+s+s1}{\PYZsq{}}\PY{p}{,} \PY{n}{pca}\PY{o}{.}\PY{n}{components\PYZus{}}\PY{p}{)}
\end{Verbatim}


    \begin{Verbatim}[commandchars=\\\{\}]
pca: [[ 1.82072365e-06  4.47584136e-01  4.41765411e-01  4.29686609e-01
   3.97979144e-01  3.69918767e-01  3.53063573e-01  6.69199565e-06
   6.94605205e-06  7.48699545e-06  7.87942983e-06  7.79690911e-06
   7.83768078e-06]
 [-3.36391191e-07  5.53642823e-01  3.91611377e-01  7.51240163e-02
  -2.67274713e-01 -4.50579053e-01 -5.09920857e-01  4.57022076e-06
  -1.29278520e-06 -7.60193355e-06 -1.05517954e-05 -1.09800464e-05
  -7.43284675e-06]
 [-1.42657603e-06 -4.49920270e-01 -3.35071473e-02  7.17866806e-01
   2.91213435e-01 -1.73749572e-01 -4.07580779e-01  5.45394684e-06
   1.85954127e-05  3.88365681e-06 -6.73941550e-06 -9.32576735e-06
  -3.24824575e-06]
 [-4.45743980e-07 -1.92812760e-01  1.68834826e-01  4.13771338e-01
  -7.06454730e-01 -1.56148804e-01  4.89541199e-01  6.18075806e-06
   5.90076772e-06 -2.24234453e-05  5.05479320e-06  2.06556973e-05
   5.94049376e-06]]

    \end{Verbatim}

    \begin{Verbatim}[commandchars=\\\{\}]
{\color{incolor}In [{\color{incolor}37}]:} \PY{k+kn}{from} \PY{n+nn}{sklearn}\PY{n+nn}{.}\PY{n+nn}{preprocessing} \PY{k}{import} \PY{n}{StandardScaler}
         \PY{n}{features} \PY{o}{=} \PY{p}{[}\PY{l+s+s1}{\PYZsq{}}\PY{l+s+s1}{log\PYZus{}LIMIT\PYZus{}BAL}\PY{l+s+s1}{\PYZsq{}}\PY{p}{,} \PY{l+s+s1}{\PYZsq{}}\PY{l+s+s1}{BILL\PYZus{}AMT1}\PY{l+s+s1}{\PYZsq{}}\PY{p}{,} \PY{l+s+s1}{\PYZsq{}}\PY{l+s+s1}{BILL\PYZus{}AMT2}\PY{l+s+s1}{\PYZsq{}}\PY{p}{,}\PY{l+s+s1}{\PYZsq{}}\PY{l+s+s1}{BILL\PYZus{}AMT3}\PY{l+s+s1}{\PYZsq{}}\PY{p}{,}\PY{l+s+s1}{\PYZsq{}}\PY{l+s+s1}{BILL\PYZus{}AMT4}\PY{l+s+s1}{\PYZsq{}}\PY{p}{,} \PY{l+s+s1}{\PYZsq{}}\PY{l+s+s1}{BILL\PYZus{}AMT5}\PY{l+s+s1}{\PYZsq{}}\PY{p}{,} \PY{l+s+s1}{\PYZsq{}}\PY{l+s+s1}{BILL\PYZus{}AMT6}\PY{l+s+s1}{\PYZsq{}}\PY{p}{,}
                         \PY{l+s+s1}{\PYZsq{}}\PY{l+s+s1}{log\PYZus{}PAY\PYZus{}AMT1}\PY{l+s+s1}{\PYZsq{}}\PY{p}{,}\PY{l+s+s1}{\PYZsq{}}\PY{l+s+s1}{log\PYZus{}PAY\PYZus{}AMT2}\PY{l+s+s1}{\PYZsq{}}\PY{p}{,} \PY{l+s+s1}{\PYZsq{}}\PY{l+s+s1}{log\PYZus{}PAY\PYZus{}AMT3}\PY{l+s+s1}{\PYZsq{}}\PY{p}{,} \PY{l+s+s1}{\PYZsq{}}\PY{l+s+s1}{log\PYZus{}PAY\PYZus{}AMT4}\PY{l+s+s1}{\PYZsq{}}\PY{p}{,} \PY{l+s+s1}{\PYZsq{}}\PY{l+s+s1}{log\PYZus{}PAY\PYZus{}AMT5}\PY{l+s+s1}{\PYZsq{}}\PY{p}{,}\PY{l+s+s1}{\PYZsq{}}\PY{l+s+s1}{log\PYZus{}PAY\PYZus{}AMT6}\PY{l+s+s1}{\PYZsq{}}\PY{p}{]}
         \PY{n}{x} \PY{o}{=} \PY{n}{df}\PY{o}{.}\PY{n}{loc}\PY{p}{[}\PY{p}{:}\PY{p}{,} \PY{n}{features}\PY{p}{]}\PY{o}{.}\PY{n}{values}
         \PY{n}{y} \PY{o}{=} \PY{n}{df}\PY{o}{.}\PY{n}{loc}\PY{p}{[}\PY{p}{:}\PY{p}{,}\PY{p}{[}\PY{l+s+s1}{\PYZsq{}}\PY{l+s+s1}{default}\PY{l+s+s1}{\PYZsq{}}\PY{p}{]}\PY{p}{]}\PY{o}{.}\PY{n}{values}
         \PY{n}{x} \PY{o}{=} \PY{n}{StandardScaler}\PY{p}{(}\PY{p}{)}\PY{o}{.}\PY{n}{fit\PYZus{}transform}\PY{p}{(}\PY{n}{x}\PY{p}{)}
         
         \PY{k+kn}{from} \PY{n+nn}{sklearn}\PY{n+nn}{.}\PY{n+nn}{decomposition} \PY{k}{import} \PY{n}{PCA}
         \PY{n}{pca} \PY{o}{=} \PY{n}{PCA}\PY{p}{(}\PY{n}{n\PYZus{}components}\PY{o}{=}\PY{l+m+mi}{7}\PY{p}{)}
         \PY{n}{principalComponents} \PY{o}{=} \PY{n}{pca}\PY{o}{.}\PY{n}{fit\PYZus{}transform}\PY{p}{(}\PY{n}{x}\PY{p}{)}
         \PY{n}{principalDf} \PY{o}{=} \PY{n}{pd}\PY{o}{.}\PY{n}{DataFrame}\PY{p}{(}\PY{n}{data} \PY{o}{=} \PY{n}{principalComponents}
                      \PY{p}{,} \PY{n}{columns} \PY{o}{=} \PY{p}{[}\PY{l+s+s1}{\PYZsq{}}\PY{l+s+s1}{principal component 1}\PY{l+s+s1}{\PYZsq{}}\PY{p}{,} \PY{l+s+s1}{\PYZsq{}}\PY{l+s+s1}{principal component 2}\PY{l+s+s1}{\PYZsq{}}\PY{p}{,} \PY{l+s+s1}{\PYZsq{}}\PY{l+s+s1}{principal component 3}\PY{l+s+s1}{\PYZsq{}}\PY{p}{,} \PY{l+s+s1}{\PYZsq{}}\PY{l+s+s1}{principal component 4}\PY{l+s+s1}{\PYZsq{}}\PY{p}{,}
                                   \PY{l+s+s1}{\PYZsq{}}\PY{l+s+s1}{principal component 5}\PY{l+s+s1}{\PYZsq{}}\PY{p}{,}\PY{l+s+s1}{\PYZsq{}}\PY{l+s+s1}{principal component 6}\PY{l+s+s1}{\PYZsq{}}\PY{p}{,} \PY{l+s+s1}{\PYZsq{}}\PY{l+s+s1}{principal component 7}\PY{l+s+s1}{\PYZsq{}} \PY{p}{]}\PY{p}{)}
         \PY{n}{principalDf}\PY{o}{.}\PY{n}{head}\PY{p}{(}\PY{p}{)}
\end{Verbatim}


\begin{Verbatim}[commandchars=\\\{\}]
{\color{outcolor}Out[{\color{outcolor}37}]:}    principal component 1  principal component 2  principal component 3  \textbackslash{}
         0              -3.500827               2.365336               1.480592   
         1              -1.857898               0.154172              -0.510818   
         2              -0.545440              -1.180344               0.045414   
         3               0.000963              -0.737854               0.860512   
         4              -0.225923              -1.794903               0.952715   
         
            principal component 4  principal component 5  principal component 6  \textbackslash{}
         0               0.362212               1.284113               0.873443   
         1               0.931328               1.141986               0.412477   
         2               0.257409               0.025700              -0.222169   
         3               0.156770               0.033502              -0.092137   
         4              -0.020841               0.571804               0.419759   
         
            principal component 7  
         0               0.331434  
         1              -1.674209  
         2               0.095423  
         3               0.117068  
         4              -0.507973  
\end{Verbatim}
            
    \begin{Verbatim}[commandchars=\\\{\}]
{\color{incolor}In [{\color{incolor}38}]:} \PY{n}{finalDf} \PY{o}{=} \PY{n}{pd}\PY{o}{.}\PY{n}{concat}\PY{p}{(}\PY{p}{[}\PY{n}{principalDf}\PY{p}{,} \PY{n}{df}\PY{p}{[}\PY{p}{[}\PY{l+s+s1}{\PYZsq{}}\PY{l+s+s1}{default}\PY{l+s+s1}{\PYZsq{}}\PY{p}{]}\PY{p}{]}\PY{p}{]}\PY{p}{,} \PY{n}{axis} \PY{o}{=} \PY{l+m+mi}{1}\PY{p}{)}
\end{Verbatim}


    \begin{Verbatim}[commandchars=\\\{\}]
{\color{incolor}In [{\color{incolor}39}]:} \PY{n}{pca}\PY{o}{.}\PY{n}{explained\PYZus{}variance\PYZus{}ratio\PYZus{}}
\end{Verbatim}


\begin{Verbatim}[commandchars=\\\{\}]
{\color{outcolor}Out[{\color{outcolor}39}]:} array([0.51460566, 0.15764278, 0.06993085, 0.05446773, 0.04516576,
                0.04217214, 0.03813215])
\end{Verbatim}
            
    \begin{Verbatim}[commandchars=\\\{\}]
{\color{incolor}In [{\color{incolor}40}]:} \PY{c+c1}{\PYZsh{}Scree plot}
         \PY{n}{plt}\PY{o}{.}\PY{n}{rcParams}\PY{p}{[}\PY{l+s+s1}{\PYZsq{}}\PY{l+s+s1}{figure.figsize}\PY{l+s+s1}{\PYZsq{}}\PY{p}{]} \PY{o}{=} \PY{p}{[}\PY{l+m+mi}{10}\PY{p}{,} \PY{l+m+mi}{5}\PY{p}{]}
         \PY{c+c1}{\PYZsh{}Calculate \PYZpc{} that each comp accounts for}
         \PY{n}{per\PYZus{}var} \PY{o}{=} \PY{n}{np}\PY{o}{.}\PY{n}{round}\PY{p}{(}\PY{n}{pca}\PY{o}{.}\PY{n}{explained\PYZus{}variance\PYZus{}ratio\PYZus{}}\PY{o}{*}\PY{l+m+mi}{100}\PY{p}{,} \PY{n}{decimals}\PY{o}{=}\PY{l+m+mi}{1}\PY{p}{)}
         \PY{c+c1}{\PYZsh{}Create labels}
         \PY{n}{labels} \PY{o}{=} \PY{p}{[}\PY{l+s+s1}{\PYZsq{}}\PY{l+s+s1}{PC}\PY{l+s+s1}{\PYZsq{}} \PY{o}{+} \PY{n+nb}{str}\PY{p}{(}\PY{n}{x}\PY{p}{)} \PY{k}{for} \PY{n}{x} \PY{o+ow}{in} \PY{n+nb}{range}\PY{p}{(}\PY{l+m+mi}{1}\PY{p}{,} \PY{n+nb}{len}\PY{p}{(}\PY{n}{per\PYZus{}var}\PY{p}{)}\PY{o}{+}\PY{l+m+mi}{1}\PY{p}{)}\PY{p}{]}
         \PY{c+c1}{\PYZsh{}Create matplotlib bar (scree) plot}
         \PY{n}{plt}\PY{o}{.}\PY{n}{bar}\PY{p}{(}\PY{n}{x}\PY{o}{=}\PY{n+nb}{range}\PY{p}{(}\PY{l+m+mi}{1}\PY{p}{,}\PY{n+nb}{len}\PY{p}{(}\PY{n}{per\PYZus{}var}\PY{p}{)}\PY{o}{+}\PY{l+m+mi}{1}\PY{p}{)}\PY{p}{,} \PY{n}{height}\PY{o}{=}\PY{n}{per\PYZus{}var}\PY{p}{,} \PY{n}{tick\PYZus{}label}\PY{o}{=}\PY{n}{labels}\PY{p}{)}
         \PY{n}{plt}\PY{o}{.}\PY{n}{ylabel}\PY{p}{(}\PY{l+s+s1}{\PYZsq{}}\PY{l+s+s1}{Percentage of Explained Variance}\PY{l+s+s1}{\PYZsq{}}\PY{p}{)}
         \PY{n}{plt}\PY{o}{.}\PY{n}{xlabel}\PY{p}{(}\PY{l+s+s1}{\PYZsq{}}\PY{l+s+s1}{Principal Component}\PY{l+s+s1}{\PYZsq{}}\PY{p}{)}
         \PY{n}{plt}\PY{o}{.}\PY{n}{title}\PY{p}{(}\PY{l+s+s1}{\PYZsq{}}\PY{l+s+s1}{Scree Plot}\PY{l+s+s1}{\PYZsq{}}\PY{p}{)}
         \PY{n}{plt}\PY{o}{.}\PY{n}{show}\PY{p}{(}\PY{p}{)}
\end{Verbatim}


    \begin{center}
    \adjustimage{max size={0.9\linewidth}{0.9\paperheight}}{panek_tanyuk_wall_davieau_lab1_files/panek_tanyuk_wall_davieau_lab1_81_0.png}
    \end{center}
    { \hspace*{\fill} \\}
    
    The Percentage of Explained Variance in the plot above tells us how much
information (variance) can be attributed to each of the principal
components. This is important as while we can convert 13 dimensional
space to 7 dimensional space, we lose some of the variance (information)
when we do this. By using the attribute explained\_variance\_ratio\_, we
can see that the first principal component contains 51.46\% of the
variance, the second principal component contains 15.76\% of the
variance and the third principal component contains 6.99\% of the
variance and so on. Together these seven components explain 92.18\% of
the variance.

    \begin{Verbatim}[commandchars=\\\{\}]
{\color{incolor}In [{\color{incolor}41}]:} \PY{n}{sns}\PY{o}{.}\PY{n}{pairplot}\PY{p}{(}\PY{n}{x\PYZus{}vars}\PY{o}{=}\PY{p}{[}\PY{l+s+s2}{\PYZdq{}}\PY{l+s+s2}{principal component 1}\PY{l+s+s2}{\PYZdq{}}\PY{p}{]}\PY{p}{,} \PY{n}{y\PYZus{}vars}\PY{o}{=}\PY{p}{[}\PY{l+s+s2}{\PYZdq{}}\PY{l+s+s2}{principal component 2}\PY{l+s+s2}{\PYZdq{}}\PY{p}{]}\PY{p}{,} \PY{n}{data}\PY{o}{=}\PY{n}{finalDf}\PY{p}{,} \PY{n}{hue}\PY{o}{=}\PY{l+s+s2}{\PYZdq{}}\PY{l+s+s2}{default}\PY{l+s+s2}{\PYZdq{}}\PY{p}{,} \PY{n}{size}\PY{o}{=}\PY{l+m+mi}{5}\PY{p}{)}
\end{Verbatim}


\begin{Verbatim}[commandchars=\\\{\}]
{\color{outcolor}Out[{\color{outcolor}41}]:} <seaborn.axisgrid.PairGrid at 0x1d457017240>
\end{Verbatim}
            
    \begin{center}
    \adjustimage{max size={0.9\linewidth}{0.9\paperheight}}{panek_tanyuk_wall_davieau_lab1_files/panek_tanyuk_wall_davieau_lab1_83_1.png}
    \end{center}
    { \hspace*{\fill} \\}
    
    Above scatterplot represents 67.21\% of the continuus features variance
just using only the first 2 principal components.


    % Add a bibliography block to the postdoc
    
    
    
    \end{document}
